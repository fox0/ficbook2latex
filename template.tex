% pdflatex
\documentclass[twoside,a5paper,12pt]{extbook}
\usepackage[T2A]{fontenc}
\usepackage[utf8]{inputenc}
\usepackage[english,russian]{babel} 

\usepackage[twoside, includehead, margin=10mm, left=25mm, bottom=20mm, headsep=3mm]{geometry}

\hfuzz=2pt



\usepackage{fancyhdr}
\pagestyle{fancy}


\usepackage{indentfirst}
% \usepackage{subfig}


% \renewcommand{\numberline}[1]{}

% \makeatletter
% \renewcommand{\@seccntformat}[1]{}
% \makeatother

% \usepackage{titlesec}
% \titleformat{\chapter}[display]{\normalfont\fontsize{14}{17}\bfseries}{}{0pt}{\fontsize{14}{17}}
% \titleformat{\section}{\normalfont\fontsize{14}{17}\bfseries}{}{0pt}{}



% \titlespacing*{\chapter}{0pt}{-30pt}{8pt}
% \titlespacing*{\section}{0pt}{*4}{*4}






\title{Практикум}
\author{DahlSq}
\date{}


\begin{document}
\maketitle

Всё пошло немного не так. Сначала Найтмэр Мун
победила, а потом... Потом она удивила Твайлайт
ещё больше.

А ведь Солнце должно быть на небе. Всегда. Что
бы там не творилось на земле.

\tableofcontents

\chapter{Новое назначение}



— Подойди ближе, не перекрикиваться же через весь зал. И вы тоже.

Твайлайт на негнущихся ногах подошла. Подруги последовали за ней.

— Так. Дай-ка я сначала кое-что объясню. Твоя магия никуда не делась, я её просто запечатала. Снять печать — секундное дело, но я в любом случае сниму не раньше, чем ты успокоишься. А ты, красавица за её спиной, даже и не пытайся, сил не хватит… Это, я так понимаю, твои друзья?

— Д-да…

— Тебе повезло… Оставьте-ка нас, девочки, наедине. Подождите там, за дверью. Это ненадолго, самое большее минут через сорок ваша подруга присоединится к вам, обещаю. И… я надеюсь, у вас хватит ума не подслушивать?

Пятеро уныло кивнули.

— Прекрасно. Никто не будет подслушивать, ведь правда? — проницательный взгляд Найтмэр Мун остановился на Пинки. Та громко сглотнула и кивнула ещё раз. — Вижу, мы поняли друг друга. Идите, вас я больше не задерживаю.

Новые подруги Твайлайт поплелись к выходу. Когда они вышли, Повелительница Ночи магией захлопнула двери и вздохнула.

— Ты вообще кто? — поинтересовалась она.

— В к-каком смысле?

— Ну, имя своё для начала назови. Если, конечно, не хочешь, чтобы я всё время обращалась к тебе «эй, ты!»

— Твайлайт Спаркл…

— Прежде такие имена носили аристократы. Сейчас тоже? — Твайлайт кивнула. — Впрочем, это неважно. Простого «Твайлайт» достаточно? Я имею в виду, при общении не на публике.

Твайлайт снова оторопело кивнула. С ней ещё собирались общаться?! И спрашивали каких-то разрешений?

— Ко мне можешь обращаться «Найт», опять же не на публике. На «ты» или «вы» — как хочешь, и лучше без «вашего высочества».

— Э… «величество»?

— Если это юмор, то за тысячу лет он сильно изменился. Скажи, Твайлайт, я похожа на ненормальную идиотку?

— Да! — тут же выпалила Твайлайт из чувства противоречия. — Н-н-н… — попытался внести коррективы инстинкт самосохранения, но его тут же подавило чувство гордости. — Не знаю… не очень, — вынужден был признать здравый смысл.

Найтмэр Мун наблюдала с откровенным интересом.

— А назови-ка мне, — полюбопытствовала она, — хотя бы две причины, по которым на этот вопрос можно ответить «да»? Рациональные причины, я имею в виду.

— Ну… э… Пафосная речь при первом появлении. И это дурацкое ржание.

— Не смогла удержаться. Как я сказала, юмор должен был измениться, но… чтобы после этой тысячелетней скукотищи даже не поиграть на ожиданиях толпы? Просто представь себя на моём месте.

— Э-э-э…

— Ладно, неважно. Давай лучше о тебе. У тебя незаурядный магический талант. Ты моей сестре кто — ученица?

— Да.

— Ты её?.. — Найтмэр Мун употребила слово, которое Твайлайт прежде никогда не слышала.

— Э… прошу прощения?

— Что, сейчас так уже не говорят? Раньше этим словом называли особо доверенного ученика, который почти как родственник.

— Ну… э… полагаю, что да.

— Понятно, кого попало она бы против меня не выпустила. Наверняка ты хочешь знать, что я сделала с твоей наставницей?

— Да!

— Ничего необратимого. Изгнала её, как и она меня.

— На Солнце?!

— Я же сказала, ничего необратимого. Ты уверена, что там в принципе можно существовать?

— Нет.

— Мне почему-то тоже так кажется. Поэтому на Луну. Там можно, я проверила. То есть, мной проверили.

— На семьсот тридцать лет?!

— О! — Найтмэр Мун шевельнула бровями. — Ты уже успела посчитать следующий благоприятный момент для бегства оттуда?

— Я заранее… — буркнула Твайлайт. — На случай, если бы сегодня всё пошлó как предполагалось.

— Правильно, кстати, посчитала. Нет, семьсот тридцать многовато будет. Вполне достаточно на пару годиков. Не могла же я оставить своё изгнание совсем без отмщения? Уверяю тебя, за первую пару лет там можно разве что немного заскучать. По себе говорю.

— Пару лет, а потом?

— Верну её назад, естественно. Мне как раз успеет надоесть управлять страной.

— А если не надоест?

Найтмэр Мун расхохоталась. Вполне, кстати, натурально, без всякого истерического ржания:

— За пару лет не надоест? Да три шанса из четырёх, что надоест раньше! А раньше надоест, раньше и верну. Я, знаешь ли, умею это делать, но особой страсти к правлению не питаю.

— Но ведь тогда… тысячу лет назад, в смысле…

— Дура была, — самокритично признала Повелительница Ночи. — Впрочем, и сестрица не умнее. Если бы она тогда просто под меня прогнулась, я бы через луну-другую сама к ней прибежала умолять о возвращении. А мы вместо этого подрались, как малые дети в песочнице. За сестру не скажу, а у меня мозги на место вроде встали. Было время подумать.

— Послушайте, Н…Найт, — осмелела Твайлайт. — Я понимаю, как глупо это звучит, но я, честно говоря, не понимаю…

— Да уж вижу. Ладно, давай напрямик. Ещё раз: с сестрой моей ничего не будет. От силы через пару лет вернётся обратно, и всё станет как было. Чтобы ты знала… в былые времена мы с ней подкалывали друг дружку так, что двухлетнее изгнание на Луну по сравнению с этим просто тьфу. Легче стало?

— Да… а я?

— А что ты? Отрекаться от неё тебя никто не заставляет. Я, честно говоря, рассчитываю на твою помощь, но принуждать ни к чему не собираюсь. Когда мы закончим этот разговор, ты будешь вправе идти куда хочешь и делать что угодно. Магию твою я тебе верну.

— Ка…кую помощь?!

— Да всякую. Я же просто не в курсе нынешней ситуации, а ты в курсе. Полагаю, если уж сестра тебе доверяет, значит и я могу. Кроме того… я умею управлять страной в нормальной ситуации, а тут сейчас, как ни крути, получился кризис. На кого ещё мне положиться, как не на лучшую ученицу сестры? Репрессий и зверств никаких не будет, сразу говорю.

— Найт… я, конечно, извиняюсь, но ведь причина этого кризиса, если уж говорить напрямик…

— Да, да, да. Причина исключительно в том, что я решила выпендриться и самую малость отплатить сестрице. Если бы я вернулась и ограничилась тем, что просто наорала на неё, ничего этого не было бы. Ты кругом права. Просто поверь, что ты на моём месте тоже не удержалась бы. Да и никто не удержался бы. А кризис как-нибудь преодолеем. Поможешь мне?

— Ещё раз извиняюсь, а можно мне хоть одну причину, чтобы согласиться?

— Легко. Посмотри на это, как на двухлетний практикум по управлению страной. Такой шанс выпадает раз в жизни, и то не каждому. Ты же знаешь мою сестру, а теперь представь, как она будет обрадована, если ты на своём месте справишься.

— А…

— А ты должна справиться. Не зря же она тебя выбрала.

— Так, — довольно твёрдо сказала Твайлайт. — У меня есть ряд вопросов.

— Задавай.

— Речь по-прежнему идёт о вечной ночи, или как?

— Или как. Эта ночь будет длинной, но отнюдь не вечной.

— Насколько длинной? Два года?

— Разумеется, нет. Я в курсе, чтó без солнечного света случится с сельским хозяйством. Точный ответ зависит от того, как быстро я научусь поднимать и опускать Солнце. У нас есть на это некоторое время, катастрофа сразу не произойдёт.

— Сколько у нас этого времени?

— На этот вопрос ты сама ответишь мне. И если есть способы продлить данное время — ты их найдёшь и применишь.

— А если не получится?

— Найти и применить?

— Нет, научиться.

— Вытащу сестру с Луны, как только это станет понятно. Для нас с тобой будет грандиозное позорище, сразу предупреждаю.

— Мне не привыкать.

— Вот как? Мне тоже. Ещё вопросы?

— Есть ещё. Найт… для вас обязательно продолжать носить эту дурацкую личину?

— Это… имеет ряд преимуществ. В ней меня боятся, и мне легче будет взять ситуацию под контроль. К тому же, это часть выпендрёжа перед сестрицей. Пусть при возвращении она «сразится» со мной, «победит» мою тёмную сторону, «освободит» меня и всё такое, а потом поймёт, что всё было дурацким фарсом.

— О.

— Повторяю свой вопрос: ты со мной?

— Ещё одно. С чего кто-то будет выполнять мои распоряжения, и что случится, если пони-будь откажется их выполнять?

— А это часть практикума. Придётся научиться приказывать так, чтобы и мысли не возникло не подчиниться. Научишься, я полагаю. В самом начале, разумеется, будешь приказывать моим именем. Вряд ли кто откажется.

— Народу придётся много чего объяснить…

— Конечно. Выступишь с речью, объяснишь. А я постою за твоей спиной и поулыбаюсь вот так.

— И верните уже, наконец, магию! Как мне предполагается работать, если без телекинеза я даже писать нормально не могу?!

Рог аликорна сверкнул, и Твайлайт почувствовала, что её будто отпустило. Привычным усилием воли она материализовала письменные принадлежности и тут же забубнила:

— Итак, первоочередные задачи. Сформулировать официальную версию с заделом на скорое возвращение принцессы Селестии. Подготовить коммюнике для прессы. Выступить с речью. Определить безопасный срок, который у нас есть для того, чтобы разобраться с Солнцем. Оценить возможные меры по продлению этого срока. Реализовать их…

\begin{center}* * *\end{center}

За закрытыми дверями нервно переглядывались и ёрзали подруги. Рэрити регулярно оповещала о том, что в меру своих способностей она не чувствует никаких фатальных возмущений магии, но это не очень успокаивало.

Наконец, двери окутались сиянием, открылись, и вышла Твайлайт. Ничего фатального и впрямь не наблюдалось: куча исписанных листов, левитируемых прямо перед носом на ходу, была для неё вполне нормальным явлением.

Фиолетовая единорожка на секунду опустила свои бумаги, обвела подруг взглядом и удовлетворённо кивнула. Снова уткнулась в записи.

— Эпплджек, — произнесла она. — Бросай всё и готовь доклад о влиянии сбоев суточного ритма на культуры сельскохозяйственного значения. В перспективе на… скажем, две недели. И чтобы не только про яблоки! У тебя есть шесть часов.

— Ва?!.. — фермерша подобрала отвисшую челюсть и почесала затылок под шляпой.

— Флаттершай. То же самое, о влиянии на сельскохозяйственных и промысловых животных. Семь часов у тебя. И обрати внимание: не милых и маленьких, а сельскохозяйственных и промысловых, поняла?

Розовогривая пегаска что-то тихонько пискнула. Твайлайт строго посмотрела на неё поверх бумаг. Флаттершай пискнула ещё раз, более утвердительно.

— Рэрити. Первая задача: подготовить соображения о применении различных материалов для дополнительного утепления теплиц. Пять часов. Вторая задача: разработать проект максимально лёгкой в производстве, дешёвой и тёплой одежды. Сутки. И когда я сказала «дешёвой», это значит — никаких украшений!

Модница издала горловой звук, закатила глаза и наладилась упасть в обморок, но получила такой взгляд, что моментально передумала.

— Рэйнбоу Дэш, отныне ты возглавляешь погодную службу Эквестрии. Все сельхозугодья накрыть низкой плотной облачностью с целью максимального сбережения тепла. Срок — вчера!

— Вау… — потрясённо выдохнула радужная пегаска.

— Пинки Пай, тебе самое сложное. Первое: добыть информацию о старых праздниках тысячелетней давности и устроить ретрофестиваль. Надо успокоить народ. Четыре дня сроку. Второе: придумать и запустить цикл весёлых анекдотов о соперничестве Королевских Сестёр. Приличных анекдотов. Неделя максимум. И третье, главное: разработать сценарий грандиозного представления о возвращении принцесс Селестии и Луны через одоление Найтмэр Мун. Магические ресурсы ничем не ограничены, от слова «совсем». Это не к спеху, но через две недели чтоб рабочий вариант уже был, доработать потом будет время.

— Оки-доки-локи… — выдавила из себя Пинки, глядя на Твайлайт большими круглыми глазами по чайному блюдцу.

— Цели ясны, задачи определены. За работу, девочки. Результаты своевременно докладывать мне в Министерство Кризисного Управления.  По возможности коротко, мне вечером ещё перед народом выступать.

— Куда докладывать?!?!

— А, ну, знаете… бывшая ратуша Понивилля. Я её реквизирую под своё ведомство.

Раздав сии ценные указания, глава новоявленного министерства покинула помещение.

— Нифигасе она нас построила… — пробормотала Эпплджек.

— Сама удивляюсь, — прокомментировала стоящая в дверях Найтмэр Мун. — Талант, однако. Кстати, про задачи и сроки — это ведь она серьёзно. Дело даже не в том, что при невыполнении вам будет плохо. Плохо будет всем, и реально плохо. Так что я бы на вашем месте времени не теряла. Ну, пойду разбираться, как выкатывать Солнце…

Повелительница Ночи направилась вслед за Твайлайт.

— А говорили, «спасать Эквестрию»… — буркнула под нос Рэйнбоу Дэш.

— Правильно говорили, — бросила через плечо Найтмэр Мун. — А чем, по-вашему, мы с вами тут сейчас будем заниматься?

\chapter{Пурга и протокол}

Шёл девятый день царствования Найтмэр Мун, и было уже ясно, что катастрофа Эквестрии не грозит. Принцессе удалось научиться управлять Солнцем, хотя некоторые проблемы оставались.

По словам Найт, техника процесса почти не отличалась от лунной, но требовала гораздо, гораздо бóльших усилий. Увы, принцесса после тысячи лет магического безделья была не в лучшей форме, хотя и превосходила силой любого обычного единорога (что совсем недавно продемонстрировала на Твайлайт).

Пока что Солнце показывалось из-за горизонта на четыре-пять часов в сутки (которые сейчас приходилось отсчитывать по механическим устройствам) и поднималось не очень высоко. Эпплджек и Флаттершай заверяли, что если так продлится не дольше шести-семи недель, то с растениями и животными ничего фатального не случится. По их словам, последствия не должны были выйти за рамки того, что могло произойти при обычном неурожайном годе, а это было делом досадным и неприятным, но не более.

Пегасы лезли из шкуры вон, старательно укутывая поля облаками, как одеялами, на тёмное время суток. Твайлайт поначалу дико паниковала: все расчёты показывали, что наличных пегасо-часов категорически не хватит, но на помощь пришло племя фестралов. Выглядели эти ребята немного страшновато, однако министерство пропаганды во главе с Пинки сработало грамотно, и настроение у населения удавалось поддерживать на вполне дружелюбном и в целом позитивном уровне.

Промышленность уже выдала утеплённую одежду детских размеров в предписанных количествах и перешла теперь к выпуску размеров взрослых.

Найт, в свою очередь, заверяла, что недели через три выйдет на пик формы, и тогда суточный ритм вернётся к привычному. Для этого она активно упражнялась в магии — «качалась», как выразилась Рэйнбоу Дэш, — а пока в порядке компенсации немного поярче раскочегарила Луну, так что долгие ночи стали довольно светлыми. Успехи были: каждый следующий день принцессе удавалось поднять светило чуть выше и удерживать его чуть дольше, чем в день предыдущий. 

Третий день этих «магических качаний» ознаменовался сюрпризом. В разгар упражнений на севере сверкнуло, жахнуло, и из какого-то соседнего плана бытия вынырнула целая небольшая страна. Твайлайт тогда впервые увидела свою новую начальницу по-настоящему испуганной, но тревога оказалась ложной.

Тамошний правитель проявил себя вполне вменяемым парнем. Явившись в виде астральной проекции, он лаконично поинтересовался, кто нынче правит Эквестрией. Получив столь же лаконичный ответ «её темнейшее высочество Найтмэр Мун», парень облегчённо вздохнул и с ходу предложил договор о дружбе и торговле, каковое предложение было восторженно принято. Проекция развеялась, через час в Кантерлот телепортировалось посольство, и теперь уже Найт облегчённо вздохнула.

Отправив ответную делегацию, Твайлайт попыталась было выяснить у начальства смысл происходящего, но добилась лишь загадочного «вот говорила же я тебе, в сохранении личины есть свои плюсы!» Потом и вовсе стало не до выяснений, ибо подготовку договора естественным образом свалили именно на Твайлайт.

В данный момент процесс подходил к своему логическому завершению.

— Тысячелетний столп Кристальной Империи! Её повелитель и верховный главнокомандующий! Хранитель Хрустального Сердца и Протектор Севера! — разорялся посол во всю свою глотку. А глотка у него была, не иначе, лужёная.

К сожалению, приглушить эту речь магией не представлялось возможным: сам будучи единорогом, посол в два счёта почувствовал бы заклинание. Такое запросто можно было расценить, как неуважение, а неуважение к дипломату есть неуважение к представляемой им державе…

Твайлайт покосилась на Найтмэр Мун. Та уловила взгляд и едва заметно страдальчески поморщилась: видимо, испытывала похожие сомнения.

— Магистр кристальной магии семнадцатой ступени посвящения! Почётный доцент Кантерлотской Школы одарённых единорогов…

При этих словах Твайлайт, мягко говоря, удивилась. Ни о чём подобном она никогда не слышала… с другой стороны, Школе было много веков, и среди её традиций вполне могло быть что-то такое. Всех традиций, как она уже давно поняла, не знали даже преподаватели.

— Член-корреспондент имперской сельскохозяйственной академии! Законодатель и прокурор имперской моды…

Сзади, где стояли советники, послышалось фырканье. Эпплджек или Флаттершай… а может, и обе сразу. Пинки в данный момент отсутствовала, а Рэрити для такого была слишком хорошо воспитана… впрочем, тут бы кто угодно фыркнул. Титулование становилось откровенно бредовым.

— Его величество могучий король Со-о-омбра Вто-о-орой…\footnote{
Кто такой Сомбра Второй? В сериале был просто Сомбра, без династического нумерования, то есть «первый этого имени». Но добавление «Первый» к имени появляется только после того, как появляется другой правитель с таким же именем; стать прижизненно «Первым» при нормальном царствовании нельзя. Так что на момент событий правит его потомок/наследник не столь уж важно в каком поколении. Наверняка ему с детства вдалбливали про ужасных Двух Сестёр, которые-изгнали-нас-в-другую-реальность — а тут какая-то незнакомая Найтмэр Мун, с которой можно попробовать договориться и подружиться. С другой стороны, Найтмэр/Луна должна хорошо помнить былые дела с Империей и Сомброй, и если всё такое начинать по новой, то она одна ситуацию явно не вытянула бы, тут есть чего пугаться.
} — провозгласил посол так, что заложило уши.

Твайлайт внутренне сжалась. Если этот тип так орал на вступительной формальной части своей речи, то что же будет сейчас, на части содержательной?!

Действительность превзошла все ожидания.

— …приносит свои глубокие извинения за то, что не смог почтить это знаменательное событие личным присутствием, — сообщил посол обыденным голосом. — Он выражает надежду, что Ваше Королевское Высочество, Ваша Светлость министр и Ваши Превосходительства советники поймут его занятость государственными делами и просит меня, как своего полномочного представителя, передать вам подписанный с нашей стороны экземпляр договора.

Он пролевитировал свиток Твайлайт, которой теперь по протоколу полагалось проверить текст (действие, разумеется, чисто формальное) и передать его на вершину власти для ответного подписания.

Твайлайт подхватила документ своей аурой, развернула перед глазами…

—  Слышь, приятель, — раздался сзади хрипловатый голос Эпплджек. — Эт’чё получается, ты тут нам всякую пургу по ушам нёс в разы дольше, чем сказал по делу?!

Её Светлость министр похолодела, буквально ощутив витающий в воздухе дипломатический скандал. Ощущения концентрировались в том самом месте, на котором обычно остаётся отпечаток высочайшего копыта в случае отставки.

Посол, однако, приятно улыбнулся:

— Ваше Превосходительство довольно точно выразили суть церемониала. Нюанс, однако, в том, что подобные речи… пургу, как вы изволили выразиться, считается абсолютно недопустимым прерывать.

Твайлайт краем уха услышала шёпот «а зря!» откуда-то из шеренги стражников и облилась холодным пóтом. Посол улыбнулся ещё шире:

— Прошу прощения, мне тут что-то послышалось…

Стража дружно сделала оловянные глаза навыкате. Эпплджек шагнула вперёд:

— Щас зато не послышится. Ты, приятель, вообще в курсе, что у нас тут как бы кризис и каждая минута на счету? Ты б своему королю-то передал, шо если он реально дружить хочет, то расфуфыриваться перед друзьями не след. Ведь половина ж того, шо ты тут гнал, пурга явная! Про академика и прокурора уж точно!

Твайлайт отчётливо осознала, что по окончании приёма кого-то будут убивать, и ей определённо придётся принять участие, причём в роли одной из жертв. Подтверждение не замедлило прозвучать:

— Нет необходимости передавать. Я имею на такой случай вполне определённые инструкции от его величества. Позвольте-ка… — посол вытащил свиток из-под носа оцепеневшей Твайлайт и телекинезом разорвал его на клочки. — Имею честь предложить вам вот это… — он пролевитировал ей другой свиток из своей сумки. — Данное предложение имеет, правда, односторонний характер, однако я вслед за его величеством не сомневаюсь, что вам придётся с ним согласиться.

— И… э… в чём же его суть? — госпожа министр развернула свиток, ёжась под тяжёлым взглядом Найтмэр Мун.

— О, текст тот же самый. Но размеры многих пошлин изменены.

— Позволено ли мне будет узнать, на чём основана ваша уверенность в нашем согласии?

— Например, на том, что других экземпляров у меня с собой нет, — улыбнулся посол.

Найт прищурилась, но тут Твайлайт воскликнула:

— Все изменения в нашу пользу?!

— Ваша Светлость верно уловили суть нашей редакции.

Просмотренный свиток был переправлен правительнице.

— Если дело обстоит именно так, — задумчиво произнесла она впервые за весь приём, — то мы, разумеется, с радостью принимаем вашу инициативу. Однако нам очень хотелось бы знать, чем она вызвана.

— У его величества… — посол чуть помолчал, подбирая слова, — своеобразное чувство юмора. Он, конечно, умеет вести дела традиционными способами, без этого нельзя быть королём, но… Отправляя меня с моей миссией, он сказал, я цитирую, «если эти ребята умеют и не боятся называть вещи своими именами, лучше дружить с ними не только на словах — выиграем куда больше, чем какие-то деньги, нам здесь жить». Конец цитаты. Полагаю, я получил нужное подтверждение.

Твайлайт прокашлялась и вновь взяла ход церемонии в свои копыта.

— Наша благодарность его величеству безмерна, — вполне искренне произнесла она, — равно как и наше восхищение его деловым подходом. Заверяю вас, мы приложим все усилия к тому, чтобы Кристальная Империя о нём не пожалела, — Найтмэр Мун с благосклонной улыбкой наклонила голову. — Однако я вынуждена извиниться: как вы понимаете, нам сейчас нужно изготовить и подписать копию договора в нынешней редакции, это займёт какое-то время.

— О, разумеется. Думаю, после всего сказанного нет никакого смысла устраивать повторную церемонию. Просто передайте нам наш экземпляр с кем-нибудь из почётного караула… например, с тем, кто позволил себе меня комментировать. В порядке небольшого наказания, воинскую дисциплину никто не отменял. Засим же позвольте откланяться…

Прощальный ритуал был исполнен в максимально упрощённом виде, посол с двумя сопровождающими отбыл, и Твайлайт тут же отправила стражника за придворным каллиграфом.

Найт вздохнула.

— Наверное, я должна извиниться, — произнесла она в пространство. — Я, правда, ничего не сказала вслух и не сделала, но про себя… В общем, я была неправа. Эпплджек, моя благодарность. Твоя… манера выражаться… сегодня принесла стране неоценимую пользу.

— Прям уж, неоценимую, — хмыкнула та. — Вон же он, свиток, в ём вся цифирь прописана.

— Я сейчас говорю не о ставках и процентах. И даже не о том, что мы, похоже, обзавелись хорошим союзником, невзирая на прошлые… гм… Я говорю вот об этом.

Принцесса протянула копыто, указуя им на Эпплджек. Точнее, на её грудь. Там на тоненькой цепочке болталась небольшая подвеска из трёх камешков: оранжевого и двух зелёных. Огранка и оправа изображали яблоко с двумя листиками.

— Эт’чё?! — удивилась фермерша. — Сроду не носила такое!

Флаттершай, Рэрити и Рэйнбоу Дэш обменялись недоуменными взглядами.

— Это… то, о чём я думаю?! — ахнула Твайлайт.

— Да, если ты думаешь об Элементах Гармонии. Элемент Честности, я полагаю.

— Вы уверены?!

— А как ты думаешь? Последний раз этой дря… этими дрягоценностями, я хотела сказать, шарахнули именно по мне. Кто и может быть уверен, так именно я!

— Селестия говорила… — пробормотала Твайлайт. — И что теперь?

— А ничего. Если история чему и учит… было бы только чем шарахнуть, а уж по кому — всегда найдётся. Авось, пригодится… хотя, лучше бы не пригодилось. Распорядись там, чтоб как копию сделают, мне на подпись принесли. Ну, и в посольство потом…

Найтмэр Мун покинула тронный зал.


\chapter{Таланты и поклонники}

На пятнадцатый день царствования Найтмэр Мун заседание Совета было расширенным и выездным. Проходило оно в Понивилле.

Как в любой стране в любые времена, провинциальный визит высоких гостей сопровождался неизбежными праз\-дни\-чно-развлекательными мероприятиями добровольно-при\-ну\-ди\-тель\-ного характера. К которым столь же неизбежно были привлечены школьники.

— …Пожалуйста, давайте поприветствуем наших юных артистов! — похоронным тоном возвестила Черили с натянутой улыбкой.

Из складок занавеса вышла Сильвер Спун в хламиде, от вида которой у Рэрити нервно задёргался глаз. Сие должно было изображать древние одеяния.

Юная пони жалобно шмыгнула носом и дрожащим голосом начала\footnote{Далее приводятся немного изменённые стихи Владимира Высоцкого — слова одной из песен, написанных им для аудиоспектакля «Алиса в Стране Чудес» (1976).}:


\begin{quote}
Злодейкой Найтмэр Мун была,
\\
Так много ела и пила,
\\
Что еле-еле проходила в двери…
\end{quote}
    
Занавес раздвинулся, и потрясённые зрители увидели злодейку в исполнении Даймонд Тиары. Сквозь её диадему была просунута свёрнутая из бумаги трубка, изображающая рог. В качестве крыльев на спину был приделан старый драный зонтик. Для соответствия образу под живот привязали здоровенный воздушный шарик, и всё это (включая саму актрису) покрасили в радикально чёрный цвет.

У Рэрити задёргался второй глаз. И не только у неё. Твайлайт плотно сжала зубы: видит небо, она приложила все усилия для того, чтобы отменить это позорище, но ей не удалось. С учётом того, \textit{какую} должность она сейчас занимала, данный факт говорил о многом.

На сцене между тем разворачивался сюжет, хорошо известный каждому со второго класса школы\footnote{Сюжет постановки навеян поэмой Владимира Маяковского
«Сказка о Пете, толстом ребёнке, и о Симе, который
тонкий» (1925)}. Когда Найтмэр Мун разнесло настолько, что она совсем перестала проходить в двери, злодейка разобиделась на весь мир и решила укутать его вечной ночью, дабы скрыть от чужих взоров свой кошмарный вид.

К несчастью для себя, она перешла все мыслимые границы. После того, как ей был подло сожран тортик, лично испечённый принцессой Селестией для сирот из приюта, её солнечное высочество преисполнилось праведного гнева и вызвало злодейку на бой.

Селестию представляли Снипс и Снэйлс. Первый сидел на спине у второго под белой попоной с криво вышитым солнцем и левитировал по бокам два веера, долженствующие изображать крылья. Принцесса получилась какой-то немного горбатой, но в общем, вполне ничего. Особенно по сравнению со злодейкой Найтмэр Мун.

Из всех участников постановки эти двое были единственными, кто явно наслаждался своей ролью. Впрочем, среди зрителей в зале наслаждавшихся было ещё меньше.

Под гробовое молчание зала сюжет плавно дошёл до своего логического завершения. Селестия нанесла Найтмэр Мун сокрушительную победу, обожравшаяся злодейка лопнула и забрызгала собой лунный диск, оставив на нём зловещий узор. Финал отыграли напрямую: Снэйлс магией лопнул шарик под животом у Даймонд Тиары, в этот момент из-за кулис сколдовали яркую вспышку, и пока зрители промаргивались, фанерный кругляш был перевёрнут с чистой стороны на запятнанную.

Настоящая Найтмэр Мун в зале радостно зааплодировала передними копытами и обвела ряды внимательным взглядом, запоминая тех немногих, кто сдуру последовал её примеру. Артисты вышли на поклон.

— Хорошие деточки, — умилённо проворковала принцесса и почему-то облизнулась. Причём в сторону Сильвер Спун и Даймонд Тиары. — Ну, бегите все учить уроки!

Закулисные постановщики из числа старшеклассников правильно поняли намёк и поторопились убраться вслед за исполнителями. А вслед за ними и представители местной администрации. 

Найт обратила взор на группу чиновников, из-за которых заседание и имело статус расширенного:

— Итак, мы полагаем, уважаемые представители Эквестрийской Образовательной Ассоциации воочию убедились в том, что школьную программу надо менять? Как мы с вами только что увидели, ряд входящих в неё произведений просто не соответствует исторической действительности. Если кто-то не согласен с нашими предложениями, мы готовы выслушать все претензии. Даже в сии кризисные времена мы не пожалеем нашей магии на исправление недоразумений, советник Нэйсэй может подтвердить. То есть уже не может, ну, вы нас понимаете.

Произошло шевеление. Предводитель чиновников добыл из портфеля документ о нескольких страницах, подписал и пустил по кругу. Принцесса пронаблюдала за подписанием и одобрительно кивнула:

— Не смеем вас далее задерживать. Прочие вопросы мы вполне решим с малым Советом, на компетентность которого более чем полагаемся… более чем на некоторых, мы имеем в виду, вы нас понимаете.

Дождавшись, когда за последним закроется дверь, Твайлайт лаконично спросила:

— Зачем?

— Надо, — столь же лаконично ответила Найт.

— Кому?

— Мне.

— Зачем?

— А не понимаешь? Или ты, когда была школьницей, сама это не учила наизусть? Пинки Пай… покажи ей!

— Ой, можно? — подпрыгнула та. — Правда можно? Я щас… ага… вот!


\begin{quote}
У министра у Твайлайт у Спаркл
\\
Приключился припадок-инфарктл
\\
От того, что принцесса…
\end{quote}
    
Стих не был дочитан — Твайлайт метнула в Пинки такой взгляд, что воздух заискрился. Буквально.

— Что, не понравилось? — ласково осведомилась Найт. — Ты теперь министр, привыкай. И как же мы себя чувствуем?

— Вовсе не обязательно было это так обставлять! Им достаточно было просто приказать, они бы построились и в два счёта всё сделали. Могли бы мне поручить, в конце концов, я бы сама…

— Ишь ты, «сама». Смелая какая… чтобы ты знала, там у них тоже были смелые. Но глупые. И ты не ответила на вопрос: как ты себя чувствуешь с этих стишков?

— Как оплёванная, — вынуждена была признать Твайлайт. — Мне же теперь это «Спаркл-инфарктл» до конца жизни будет помниться.

— Ну и какие тебе ещё нужны объяснения?

— Такие! Детей-то зачем в это впутывать?

— Ну, во-первых, это ведь детей и касается. Наверняка мы видели ещё далеко не самый худший вариант. А во-вторых, я позаботилась, чтобы впутанными оказались не абы какие дети.

— Это… значит, вот так, да? — Твайлайт посмотрела на Эпплджек. На Рэрити. На Рэйнбоу Дэш.

— Именно так.

— Послушайте, Найт! Мне эти трое жеребят… они ещё называют себя Меткоискателями… самóй очень симпатичны! Но вам не кажется, что решать их школьные проблемы — это не государственное дело?

— Не государственное? Вас у меня шестеро. На вас всё держится, и сколько вы обычно отдыхаете? Четыре часа в сутки, пять? И когда половина из вас половину этого времени убивает на утешение сестёр из-за этой мелкой школьной дряни — это тебе не государственное?

— Но…

— А потом эти трое «симпатичных тебе» ложатся спать, и им эта дрянь снится, а эти сны ещё и на меня валятся! Знаешь хоть, каково это? Я, между прочим, не больше вашего отдыхаю, и если ты думаешь, что разгребать эти кошмары доставляет мне удовольствие, то сильно ошибаешься!

— Но можно же было как-то по-другому это сделать, даже если так! Хотя бы спросить мнения тех, кто заинтересован! Эпплджек?!

— Ну… — фермерша почесала шляпу. — Врать не буду, я тем двум поганкам ремня-то выписала бы с удовольствием, шоб им недели две потом не сиделось. Да только не при всех же. Никого не след так-то позорить, вот…

— Рэрити?!

— Это было впечатляюще, безусловно. Очень. Но совершенно безвкусно. Я бы сказала… тут как ещё посмотреть, кто больше опозорился…

— Дэш?!

— Круто, конечно. А только крутость показывать против тех кто сам не крут, это как-то не очень круто.

— Вот! Видите?!

— И что? Могу повторить для непонятливых: так было надо.

— А можно для особо непонятливых ещё и объяснить, зачем?

Найтмэр Мун вздохнула.

— Самое время для уроков… Ну ладно. Когда ты отправлялась через лес к Замку Двух Сестёр изгонять меня, ты взяла с собой подруг. Зачем?

— Они сами пошли со мной. Помогать мне. Как же ещё? — недоуменно ответила Твайлайт.

— Уточняю вопрос. Зачем ты взяла с собой всех подруг? Могу понять про Эпплджек и Рэйнбоу Дэш. Это сила, смелость, разведка, поддержка с воздуха. Пинки Пай… ну, допустим. А Рэрити? При всём к ней уважении, она обычный средний маг совершенно не боевого толка. А Флаттершай?!

— Ну… Я что, должна была кому-то сказать «от тебя там толку всё равно не будет, сиди лучше здесь и не путайся под копытами»? Помощь и поддержка лишними не бывают… в конце концов, я просто чувствовала, что так будет правильно.

— О-о-о! — Найт ехидно заулыбалась. — «Я чувствовала, что так будет правильно»? Хорошо запомни эту фразу, я тебе буду её время от времени цитировать. Собственно, получай: я устроила это показательное позорище именно потому, что чувствовала то же самое. Удовлетворена?

 — Нет! — радостно воскликнула Твайлайт. — Если что, наш поход, основанный на этом чувстве, окончился поражением. Плохое сравнение!
 
 — Паршивая я наставница… — сообщила Найтмэр Мун в потолок. — Можно подумать, меньшим составом вы бы победили. Ну, раз уж об этом зашла речь… Почему вы проиграли?
 
 — Вы сильнее и гораздо опытнее, это же очевидно.
 
 — Плохой ответ, на «троечку». Его можно было бы принять, если бы вы просто проходили мимо и заглянули в развалины. Но вас послала моя сестра, а она ещё опытнее меня. Чем, по-твоему, она думала, поручая тебе эту миссию и надеясь на успех? Ученики твоего уровня… это, знаешь ли, отнюдь не расходный материал.
 
 — Э-э-э… — Твайлайт не задумывалась об этом с такой позиции только потому, что с момента поражения у неё не было времени. Но теперь, когда её буквально ткнули носом…
 
 — Спорить готова, когда она отправляла тебя сюда, то говорила странные вещи. Попробуй припомнить, что показалось тебе самым странным?
 
 — Да всё! Я ей о жутком пророчестве, а она меня чуть ли не пинком в провинциальный городок! Праздник праздновать и с друзьями дружиться!..
 
 — Стоп! Чуть позже ты поймёшь, а пока просто прими, что сестра всё сделала правильно и ты всё делала правильно. Так почему вы проиграли?
 
 — Ну… э… если с тех двух сторон всё делалось правильно… Это звучит по-дурацки, но видимо, мы столкнулись с какой-то неправильной Найтмэр Мун. Ну да! Помнится, я ведь именно так и подумала.
 
 — В самую точку. Теперь подумай ещё. Тот зáмок был средоточием моей силы, ну и сестры, конечно. Наивно было бы полагать, что у меня нет способов почувствовать приближение… хм… гостей вроде вас. Что сделала бы «правильная» Найтмэр Мун?
 
 — Наверное, помешала бы нам всеми силами. Постаралась бы угробить нас ещё на подступах.
 
 — Конечно. И тем самым дала бы вам кучу возможностей проявить свои лучшие качества и обрести Элементы Гармонии, обретавшиеся когда-то в замке. Шестеро совсем-совсем разных пони… совсем уж очевидный намёк.  Будь у вас хотя бы два Элемента из шести, и бой был бы уже на равных. А против трёх я бы без вариантов проиграла. Поэтому я предпочла спокойно дождаться вашего прихода и попробовать договориться по-хорошему. Ну, относительно, конечно. Получилось.
 
 Твайлайт на несколько секунд застыла с отвисшей челюстью. Кусочки головоломки, о которых она до сих пор даже не задумывалась, сложились в единое целое, и теперь это требовало осмысления.
 
 Найт хмыкнула и переключила внимание на её подруг.
 
 — А вы, красавицы? Когда Твайлайт вышла к вам и начала строить… почему вы бросились выполнять? Вас никто не пугал и никто на вас не давил.
 
 Четыре взгляда сошлись на Рэрити, которая после Твайлайт обладала самым хорошо подвешенным и при этом наиболее воздержанным языком. Та пожала плечами и озвучила:
 
 — Твайлайт, она… коротко говоря, умная и образованная. А я, конечно, маг средненький, но если бы рядом со мной кому-то промыли мозги, то я бы наверняка почувствовала. Не было такого. И магия снова была при ней. А все эти задания и сроки… сначала показались немного шокирующими, но на второй взгляд уже довольно разумными. Благо меня озадачили не в первую очередь. В общем, у меня было впечатление, что всё это действительно нужно.
 
 — «Было впечатление, что так нужно…» Ничего не напоминает?
 
 — Ой.
 
 — Вот именно. Запомните, девочки, «я считаю, что так нужно» — это важнейший и зачастую единственный инструмент правителя. Что особенно важно, он работает даже тогда, когда остальные так не считают… у хорошего правителя, конечно.
 
 — Круто! — прокомментировала Рэйнбоу Дэш. — А если потом того… окажется, что нужно было не так? Или что так было совсем не нужно?
 
 — Хороший вопрос. Тут дело вот в чём. Если правитель что-то там считает, и потом оказывается, что он считал это правильно, то всё приписывают его прозорливости, гениальности и так далее. Это хорошо запоминается. А если он немножко чересчур часто ошибается… то довольно быстро перестаёт быть правителем. А в запущенных случаях просто перестаёт быть.
 
 — КАК. ВАМ. НЕ СТЫДНО?!
 
Это было произнесено чуть громче шёпота, но в исполнении Флаттершай звучало практически как крик. Все поперхнулись и уставились на неё удивлёнными глазами. Жёлтая пегаска вспорхнула на стол, встала лицом к Найт и вперила в неё свой фирменный взгляд.

— По-моему, тут кое-кто уводит разговор от темы! Если кое-кто забыл, там остались двое обиженных и испуганных детей! Про которых даже заинтересованные стороны сказали, что это перебор! Кое-кому следовало бы перед ними извиниться, успокоить их и объяснить им по-хорошему, а иначе…

— А иначе что? — с явным интересом осведомилась Найтмэр Мун.

— Иначе моё мнение об этом кое-ком изменится в худшую сторону… — смущённо пискнула Флаттершай и спрыгнула со стола на своё место.

— Извиняться, успокаивать, объяснять? И не подумаю. У меня есть идея получше. Полагаю, будет вполне достаточно, если с ними в своей неповторимой манере поговорит Элемент Честности… — принцесса кивнула на Эпплджек. — Скажет им то самое, что сказала нам тут… я имею в виду, насчёт ремня так, чтоб две недели не сиделось. Прокомментирует, что лично я с этим вполне согласна и что её власти придворного советника для такой процедуры более чем достаточно даже без высочайшего одобрения. А потом добавит, что нашлась лишь одна-единственная пони, которая за этих двух дур по-настоящему заступилась… и пояснит, что даже изгнанная на тысячу лет злодейка Найтмэр Мун имеет какое-никакое представление о доброте… в отличие от некоторых. Все удовлетворены?

Эпплджек хмыкнула и кивнула, а вслед за ней и Рэрити с Рэйнбоу Дэш. Взгляд принцессы задержался на Флаттершай… та пошевелила губами, но ничего не сказала и тоже кивнула.

— Прекрасно. Между нами говоря, а главное-то доказательство того, что всё это было не зря, никто из вас и не заметил.

— Какое ещё доказательство? — подозрительно поинтересовалась Твайлайт.

— Да вот же оно сидит. Элемент Доброты собственной персоной, прошу любить и жаловать.

На шее Флаттершай висела подвеска с розовым камнем, огранённым в форме бабочки.

— Хотя наставница из меня всё-таки паршивая…— самокритично вздохнула Найт.


\chapter{Черновик и реальность}

Дверь скрипнула.

— Разрешите доложить, Ваша Светлость…

— Да, лейтенант? — Твайлайт оторвалась от своего черновика.

— Вы изволили приказать сообщить через полчаса.

— А что, уже? — взгляд на часы. — Да, действительно… И что?

Ответ, в общем-то, вполне читался на физиономии гвардейца. Тот совсем не по-военному вздохнул и пожал плечами.

— Без изменений? Так и продолжает?

— Никак нет, изменений не заметил. Так точно, продолжает.

— Принято, спасибо. Знаете что, лейтенант… просто сообщите мне, когда ситуация изменится. Чтобы зря не бегать и не отвлекаться.

— Есть, мэм! — дверь аккуратно закрылась с той стороны. Твайлайт тоже вздохнула и вернулась к бумагам.

Ситуация — не та, о которой был сейчас разговор с лейтенантом, а глобальная — становилась какой-то неопределённой. Письмо на столе подтверждало, что это начали замечать и другие.

Аккуратно вычерчиваемый график показывал, что за последнюю неделю продолжительность дня не увеличивалась и даже напротив — уменьшилась на две минуты. Найт, будучи спрошенной, довольно ясно и убедительно дала понять, что в разговорах с ней затрагивать эту тему пока не следует.

Однако такие графики чертила не только Твайлайт. Сомбра прислал с севера семена и саженцы сортов, специально выведенных имперскими агрономами для условий полярной ночи — безвозмездно, да ещё предложил помощь консультантами и рабочей силой. Эпплджек сейчас как раз занималась тем, что принимала и распределяла подарок.

Между обтекаемо-вежливых строк сопровождающего письма явственно читалось, что король исходит из нынешнего суточного ритма, не питает никаких иллюзий насчёт выживания Кристальной Империи в одиночку, а потому готов пособить соседям всем, чем только может, и просит лишь не забывать об этом, если помощь понадобится его стране. Точнее, когда понадобится.

В настоящее время продолжительность дня примерно соответствовала поздней осени: чуть меньше восьми часов. Не так уж страшно, поскольку за неудачным праздником Солнца всё-таки осталась половина лета… не так уж страшно на этот год. Запасы на случай неурожая были. А вот потом…

Попавший в такую же ситуацию Сомбра, конечно, поступил умно, правильно и замечательно. За сделанный им подарок следовало хвататься всеми четырьмя копытами, а также телекинезом и крыльями (у кого они были), однако…

Однако честный ответ предполагал хоть какую-то информацию о причинах происходящего и прогноз на будущее. Твайлайт не побоялась бы сунуться к Найт повторно… собственно, она и собиралась — в конце концов, речь сейчас шла не о её личном любопытстве, а о добрых отношениях с соседями! — но Найт в настоящий момент была выведена из строя и никакой информации дать не могла.

А ведь этот день начинался так обыденно! Можно даже сказать, скучно.

***

Сначала… ну да, сначала было письмо Эпплджек из Ванхувера. Груз от Кристальной Империи благополучно прибыл, получено столько-то того, столько-то сего, приступаем к распределению, просьба прикомандировать полтора десятка быстрых курьеров, всё такое…

Твайлайт озадачила Рэйнбоу Дэш насчёт курьеров, отправила Рэрити в посольство согласовать вопрос насчёт консультантов (со своими рабочими копытами в Эквестрии, хвала небесам, проблем пока не было) и села за ответное письмо к его величеству Сомбре.

Стандартно-формальные строки о безмерной благодарности с заверениями готовности оказать всяческую ответную помощь (между которых читалось полное согласие в оценке ситуации) родились довольно быстро. Дальше нужно было сообщить что-то по сути дела, и Твайлайт начала прикидывать список вопросов для Найт параллельно с формулировкой, в которую можно было бы упаковать ответы…

Вот тут-то в дверь и постучали.

— Виноват, мэм! — громогласно озвучил очевидный факт гвардеец в броне с лейтенантскими шевронами.

— Вы даже не представляете, насколько… Что там у вас случилось, офицер?

— Осмелюсь доложить, мэм, ситуация выходит из-под контроля.

Брови Твайлайт взметнулись вверх. Никакого шума она не слышала, никаких возмущений в магии не чувствовала…

— Раз уж осмелились, доложите подробности.

— Виноват, мэм, испытываю затруднения. Не знаю, как лучше описать происходящее, и не могу знать причину…

Брови госпожи министра поднялись ещё выше.

— Лейтенант, — задумчиво произнесла она. — Я полагала, что военные училища должны быстро избавлять кадетов от привычки к употреблению выражений типа «не знаю, как описать происходящее»…

— Так точно, избавляют!

— В таком случае, не доводилось ли вам слышать о понятии «неполное служебное соответствие»?

— Так точно, доводилось! Осмелюсь доложить, мэм, проблема не во мне и не в соответствии. Проблема вокруг. Если Ваша Светлость соизволят подойти к окну…

Её Светлость соизволили.

По изумрудному небу медленно проплывал косяк творожных сенбургеров. На фиолетовом газоне радостно скакали гигантские осьминоговые квакузнечики. С ветки брыкучей пивы задушевно мурлыкали часы с кукушкой, висящие там вверх ногами… причём «вверх ногами» отнюдь не было фигурой речи, у \textit{этих} часов ноги действительно имелись.

— И хрюкотали зелюки, как мюмзики в мове…\footnote{Твайлайт цитирует стихотворение Льюиса Кэрролла «Бармаглот» (1855) в переводе Дины Орловской.} — задумчиво проговорила Твайлайт. — Должна извиниться перед вами, офицер, теперь я понимаю суть ваших затруднений.

— Хрюко… как вы сказали? Это вот так называется?

— Неважно. Теперь, когда я поняла суть происходящего, расскажите предысторию.

— Предыстория, мэм, тоже не особенно понятная. Ежели Ваша Светлость знают дворцовый сад…

— Знаю.

— Там есть лужайка со статуями, и на ней такая… Виноват, запамятовал название. Статуя изображала такого скрюченного типа… ну, от каждой твари по кусочку, если вы понимаете, о чём я.

— Да, я поняла. Но вы сказали, «изображала»?

— Так точно, в этом вся и суть. В общем, статуя ожила, этот тип сошёл с постамента и… собственно, вот. Сами изволите видеть.

— Вижу. И как же он, собственно… это «вот»? Я не чувствовала и не чувствую ничего магического.

— Не могу знать, мэм. Флэш Се… один парень из моего взвода употребил выражение «гнёт реальность об колено».

— Хм… об колено? Довольно точно сказано. Знаете, лейтенант, по-моему, вы сейчас обратились не по адресу.

— Осмелюсь напомнить, мэм, с восхода Солнца прошёл всего час. Не велено беспокоить в пределах полутора часов до и после восхода и заката.

— Я помню, но речь не об этом. Прежде чем беспокоить Её Высочество… найдите советника Пинки Пай. Скажите, что вас послала я, расскажите всё то же самое, что сейчас рассказали мне. Дальше действуйте по обстоятельствам и согласно её распоряжениям. Вопросы есть?

— Так точно. Почему именно Её Превосходительство?

— Потому что, лейтенант, скажу вам честно, лично я гнуть реальность об колено не умею. И не знаю никого другого, кто умел бы. Если понадобится, обращайтесь прямо к Её Высочеству. Можете игнорировать распоряжение и ссылаться на меня. Выполняйте.

— Есть, мэм!

Гвардеец исчез за дверью. Твайлайт вернулась к столу и стала собирать в голове витиеватую сложносочинённую формулировку, которая уже почти совсем сложилась, но рассыпалась из-за этого неожиданного доклада.

После определённых умственных усилий фразы наконец выстроились, первая из них начала ложиться на бумагу… и тут снова открылась дверь. На этот раз в кабинет вошла Пинки.

Правда, это была какая-то не совсем обычная Пинки. Она и вошла-то нога за ногу, чего за ней обычно сроду не водилось. И голос у неё был какой-то негромко-неуверенный:

— Твай?.. Я там это… в общем, если что, пусть у меня из зарплаты вычитают, ладно?

Твайлайт не на шутку встревожилась:

— Пинки, да что там у вас такое?! Опять, что ли, ситуация из-под контроля выходит?

— Если бы. Я его… этого контроля то есть… слишком много себе позволила, вот.

— Да можешь ты нормальным языком объяснить-то, в конце концов?!

— Ну «нормальный» тут не совсем то слово… — Пинки чуть оживилась. — В общем, прибежал этот лейтенант, сказал, что ты велела приглядеть за реальностью, я и пошла. Там в саду этот тип, ожившая статуя, ну ты в курсе. Попробовала с ним поговорить, он таким шутником оказался! Слово за слово… ну поприкалывались мы друг над другом. У него шутки немного старомодные, но классные, моей сестре такие нравятся, я даже парочку запомнила, надо будет потом записать…

— Пинки!

— …то есть, я имею в виду, как время будет. Вот. А потом я ему прикольный стих прочитала. Тот самый, из-за которого Флатти давеча строила Найт, помнишь? Он хохотал как псих, а потом спросил меня, кто такая эта Найтмэр Мун. Мне бы тогда сразу сообразить, что статуя-то должна быть древнющей, ведь это ж тысячу лет назад было, и если он не знает, то выходит…

— Пинки!!!

— Ну, в общем, я не сообразила сразу и ответила ему как есть. Что Найтмэр Мун — это вроде как принцесса Луна, хотя и не совсем. А он тогда… вот.

— Да что «вот»?! Я сегодня с ума сойду от ваших описаний и докладов!..

— Ну… — Пинки повозила копытом по полу. — В общем, он тоже лопнул. От смеха. Такая яркая вспышка пыхнула, а когда я снова видеть смогла, вокруг меня от него только каменные куски валялись. Кстати, можно я парочку сестре отошлю? Она как раз камни изучает, я тебе рассказывала про неё…

— Пинки. Я сейчас сама лопну. Или на куски рассыплюсь. Я уже даже не знаю, что сейчас сделаю.

— Прости, увлеклась. Так вот, он лопнул, тут я про тысячу лет и сообразила. Струхнула маленько. Если эта статуя такая древняя, то она же ценная, наверно? В общем, в случае чего пусть из моей зарплаты вычитают, я согласна.

— Да в Тартар статуи с зарплатами! С реальностью-то что?!

— А, это… Ну, как он лопнул, так вроде всё в норму вернулось. Можешь сама убедиться.

Твайлайт встала из-за стола, подошла к окну и убедилась. Глянула на часы:

— Иди, докладывай Найт. Полтора часа с восхода уже прошло, теперь можно.

— А она меня не того…

— Думаю, что нет. В конце концов, реальность важнее.

— Оки-доки!

Пинки Пай уже в своём нормальном состоянии упрыгала за дверь. Твайлайт вернулась за стол и снова заскрипела пером по бумаге.

Всего лишь каких-то пять минут — и дверь распахнулась так, будто её пытались вышибить пинком. А может, и пытались. 

В кабинет влетел всё тот же лейтенант. За его спиной виднелась испуганная Пинки.

— Виноват, Ваша Светлость! Осмелюсь доложить, контроль над ситуацией опять утерян! Как бы это сказать-то…

Твайлайт подавила сильное желание запустить в гвардейца чернильницей. Глянула в окно, но там всё было нормально.

— Ведите, — коротко велела она, выходя из-за стола.

Лейтенант шустро порысил по коридору, Твайлайт и Пинки за ним. Вид у Пинки был довольно бледный, так что расспрашивать её сейчас не очень хотелось; проще было посмотреть своими глазами.

Через минуту у госпожи министра появилось смутное подозрение, а ещё через пару минут оно превратилось в уверенность. Лейтенант привёл её не куда-нибудь, а к дверям кабинета Найтмэр Мун.  

Перед коими вместо одного стояло аж трое стражников-единорогов, зачем-то державших мощный «полог тишины». Увидев высокое начальство, они чётко отсалютовали и расступились. Один из них распахнул дверь.

Твайлайт заглянула внутрь и обомлела.

Её темнейшее высочество Найтмэр Мун изволили валяться на спине посреди кабинета, раскинув во все стороны ноги-крылья, и громко, неприлично ржать, неконтролируемо болтая в воздухе упомянутыми ногами и крыльями.

Быстрый взгляд на принцессу. Быстрый взгляд на Пинки. В голове сложились два и два.

— Ты, видимо, успела ей всё рассказать?

— Ага.

— А она вот так?

— Угу.

— Смеховая истерика, — тоном знатока констатировал лейтенант. — Вообще-то при ней рекомендуется хорошая плюха… — Твайлайт задумчиво посмотрела на него, — …но присяга не позволяет.

— Значит, так. Никого, кроме меня, не пускать. «Полог тишины» продолжайте держать, это одобряю. Через полчаса заглянуть внутрь и доложить мне ситуацию. Истерики вечно не длятся.

— Есть, мэм!

\begin{center}* * *\end{center}

Твайлайт встряхнула головой, отгоняя утренние воспоминания. Хм, «утренние»… Всего полчаса, как это кончилось, и… собственно, ещё даже не кончилось. Истерики, конечно, не длятся вечно, но сколько может длиться истерика у вечного аликорна…

У неё было стойкое ощущение, что если продолжить работу над черновиком, то произойдёт ещё чего-нибудь. Поэтому она ограничилась тем, что переписала его в нынешнем состоянии целиком с учётом всех правок.

В дверь опять постучали.

— Входите, лейтенант…

На сей раз физиономия гвардейца лучилась радостью.

— Что, перестала?

— Так точно, вашсветлость! Сорок минут общим счётом.

— Спасибо, иду. Снимите усиленную охрану у дверей и… пошлите кого-нибудь за советником Пинки Пай.

— Снято и послано, осмелюсь доложить!

— Благодарю за службу, лейтенант.

— Рад стараться, Ваша Светлость!

Пинки Пай уже переминалась с ноги на ногу у двери начальственного кабинета; она подошла минутой ранее и заходить одна явно побаивалась.

Твайлайт собралась с мыслями и толкнула дверь.

— А, это вы… — вздохнула Найт за своим столом. — Да.

— Что «да»?

— Да, я уже в порядке. Вы же ведь это хотели спросить.

— Не только. Найт, вам не кажется, что вы… э… задолжали мне несколько объяснений?

— Возможно. Ну, спрашивай. Впрочем, следующий вопрос я тоже знаю. Это был Дискорд.

На лице у Твайлайт, очевидно, что-то отразилось. Потому что Найт удивлённо шевельнула бровями:

— Хочешь сказать, сестра тебе про него ничего не говорила?

— Н-нет… то есть… Я вроде бы пару раз где-то встречала или слышала это имя, но без всяких подробностей.

— Не самый умный шаг со стороны сестрицы. Хотя… там и подробностей-то тех…

— Что же это за подробности?

— Четыре слова: почти всемогущий дух Хаоса.

— Хаоса?.. Да, это было довольно похоже. Но я ничего не чувствовала на магическом уровне!

— И не должна была. То, что творит Дискорд, это не магия. Во всяком случае, совсем не такая магия, как у нас. Большего сказать не могу, просто не знаю. Возможно, сестра чего и наисследовала в моё отсутствие, да и то вряд ли.

— А при чём здесь эта статуя в саду?

— Мы с сестрой заточили его в камень за сто с небольшим лет до того, как меня… бросили на периферию. Возле замка поставили, чтоб присматривать проще было. Потом она, очевидно, перетащила сюда из тех же соображений.

— И как же вам удалось его заточить, если он такой всемогущий?

— Хороший вопрос. Для этого понадобилась вся наша объединённая сила, да ещё сила всех Элементов Гармонии. Сама понимаешь, такому трудно что-то противопоставить.

— Однако сегодня он освободился.

— Совсем хороший вопрос. У меня есть теория… она мне не нравится, но неплохо кое-что объясняет. Древние заклинания со временем слабеют. Похоже, моя сестра поддерживала их по мере надобности, но теперь её нет. А я не в полной силе, да и в курс дела пока не вошла до конца.

— Вы сказали, заклинания?

— А ты молодец.

— Значит ли это, что Кристальная Импе…

— Да. Сомбра Первый… хотя тогда он, конечно, «первым» не назывался… был тем ещё типом. Нам очень повезло с его наследником.

— И с тем, что у нас есть Эпплджек.

— И с ней тоже. А ещё с этой моей личиной.

— Да, вы говорили… хотя я так и не поняла, при чём тут это.

— Там, на Севере, должны хорошо помнить Двух Сестёр… не с самой приятной для северян стороны. А под этим именем и в этом виде они меня не знают.

— У вас была бурная молодость.

— Бурная? Не то слово. Впрочем, тебе тоже будет что вспомнить на склоне лет. Ещё вопросы есть?

— Есть. Очевидно, мы теперь можем не бояться Кристальную Империю. А Дискорда?

— Не можем. Для него лопнуть как… ну, как для пони сменить подковы. Если сходишь в сад, увидишь статую на прежнем месте, хоть и в другой позе.

— Что делать, если опять оживёт?

— Во-первых, это будет не завтра и не послезавтра. Я там… как прохохоталась, первым делом кое-что предприняла, исходя из своей теории. А во-вторых, пусть оживает. У нас уже сейчас есть половина Элементов Гармонии.

— Треть.

— Половина. Элемент Смеха только что воплотился, вон, посмотри сама.

Подвеска Пинки была голубым камнем, огранённым в форме воздушного шарика. Найт вздохнула:

— Хотела бы я ещё знать, в ответ на что он воплотился…

— То есть как?

— А вот так. Пинки Пай могла получить его одним из двух способов. Либо за то, что лопнула смехом Дискорда… либо за то, чтó сделала со мной. Выбить меня на сорок минут — тоже достижение не из мелких.

— А что, есть разница?

— Теперь уже нет, просто любопытно. У вас ко мне всё?

— Нет. Ещё чего-нибудь такого можно ожидать… из вашей молодости?

— При мне такого больше не было. Но сама понимаешь, если сестра что-то подобное учинила без меня в одиночку, то я этого знать уже не могу.

— Возвращаясь к теме Кристальной Империи… там у меня лежит письмо Сомбры. В котором он намекает, что ему очень хотелось бы иметь какую-то определённость относительно суточного ритма. Я помню тот наш разговор, но речь не обо мне.

— Суточного ритма… — принцесса поморщилась. — Ты там что-нибудь уже успела набросать для ответа?

— Черновик готов процентов на восемьдесят. Мне в это утро не очень-то дали поработать.

— Перебросишь его мне сюда вместе с тем письмом. Сама отвечу. Всё, девочки, утро кончилось, работаем!

\chapter{Дань}

Рэрити в последний раз оглядела гостевые апартаменты и сама себе удовлетворённо кивнула. После чего сразу же огорчённо вздохнула. Приятно, конечно, когда твоему вкусу и чувству изящного доверяют настолько, что именно тебе поручают организовать размещение высокого гостя… Но когда такое происходит в самый последний момент и из-за этого приходится опаздывать — подобное приятным отнюдь не назовёшь.

Впрочем, в данном случае всё удачно свелось к тому, что она вошла последней; участники встречи ещё только занимали свои места.

Сегодняшняя гостья выглядела странно: высокое худощавое нескладное существо, которое в принципе могло бы называться аликорном, если бы не стрекозиные крылья. Из-за чёрной масти и гривы ядовито-зелёного цвета общий вид был довольно мрачным, да и прочих странностей хватало. Необычные глаза с двойной радужкой и щелястыми зрачками… но самое большое удивление вызывала какая-то повсеместная дырчатость и зазубренность.

Судя по общему недоумению, никто из подруг раньше подобных существ не видел. Твайлайт откровенно пожалела об отсутствии Найт, которая вместе с Рэйнбоу Дэш должна была вернуться только послезавтра, но вести переговоры было нужно — и нужно было именно ей.

— Прежде всего, как нам следует к вам обращаться?

— Меня зовут Кризалис, и я королева роя чейнджлингов. Так что — «Королева», «Кризалис», или «Королева Кризалис».

— «Королева», то есть «Ваше Величество»?

— У нас нет титулов, как у вас, даже «королева» — это всего лишь занимаемое в рое место. При общении через коллективный разум позиции собеседников и их отношение друг к другу предельно ясны, так что в титулах нет необходимости. Но если вам так удобнее…

— Меня и занимаемое мной место вы, очевидно, знаете. Но если так, для вас не должно быть неожиданностью и то, что я заняла его несколько внезапно. Как следствие, для меня многое оказалось в новинку, и… э…

— Вы пытаетесь максимально вежливо сказать, что никогда раньше не слышали о нашем народе, — усмехнулась Кризалис.

— Да, — с некоторым облегчением призналась Твайлайт.

— Ничего удивительного, о премьер-министре Спаркл и её советниках всего несколько недель назад тоже никто не слышал. Наверное, главное вы уже поняли. Мы довольно близки к вам, пони, но в нас есть очень многое от насекомых… — королева зачем-то встала из-за стола и чуть отошла. — Вот так выглядит чейнджлинг-рядовой.

Кризалис окуталась зелёным сиянием, а когда оно угасло, на её месте стояло очень похожее, но чуть другое существо. Не такое крупное, довольно приземистое и из-за этого более пропорциональное.

— Рядовой?

— Рядовой, солдат, рабочий, юнит… Пожалуйста, придумайте своё слово, если хотите, просто «особь универсального назначения с подавленной репродуктивной функцией» звучит слишком длинно.

— То, что вы сейчас продемонстрировали…

— Да, это наша главная особенность. Мы можем принимать любую форму, некоторые ограничения накладывает лишь размер… — Кризалис снова окуталась зелёным сиянием и на несколько секунд превратилась в копию Твайлайт, потом опять стала собой и вернулась за стол. — Осталось добавить, что все чейнджлинги владеют магией и умеют летать. Хуже, чем ваши единороги и пегасы соответственно, но зато все.

— Вы интересный народ.

— Спасибо.

— Вас и пони… связывали какие-нибудь отношения до того, как я заняла свой пост?

— О да. У нас с Селестией было что-то вроде нейтралитета. Мы торговали с пони, принимая для удобства соответствующую форму, но никаких иных дел не вели. Пару раз помогали друг другу, но оба раза это оговаривалось особо и никаких далеко идущих последствий не имело.

— То есть вы были лично знакомы с Селестией. А с её сестрой?

— Нет, конечно. Мне не исполнилось ещё и трёх веков, так что откуда? Наши королевы живут довольно долго, но мы не бессмертны.

— Как вы понимаете, в конечном итоге вам придётся знакомиться и беседовать с Её Высочеством.

— Да, скорее всего. Но сейчас я хотела поговорить именно с вами, оттого и торопилась.

— Почему же со мной?

— А разве не очевидно? Вы гораздо менее опытны, и это даёт мне преимущество.

— Преимущество в чём?

— В переговорах.

— Но зачем оно вам? Разве вы пришли говорить не о сохранении нейтралитета?

— Ну конечно же, нет! Я пришла требовать для моего народа большего.

— Вас не устраивает нейтралитет?

— Устраивал, пока у вас не поменялась власть.

— А что при этом поменялось для вашего народа?

— Как что? Ваши позиции ослаблены, вы столкнулись с проблемами. Для нас появился шанс извлечь из этого выгоду, и мы её извлечём.

— Не лучше ли договориться о дружбе и помощи? В будущем ведь может произойти и обратное.

— Тогда и будем рассматривать этот вопрос.

— Итак, правильно ли мы вас поняли? Вы пришли заявить, что намерены воспользоваться нашим усложнившимся положением к своей выгоде? И, очевидно, выдвинуть какие-то требования?

— Именно так.

— Такое заявление очень похоже на ультиматум.

— Пока это лишь декларация. Предупреждая возможные вопросы… я очень неудобный заложник. Коллективный разум даст мне возможность видеть глазами моих детей и руководить ими ничуть не хуже, чем если бы я находилась среди них. А уничтожив меня, вы получите войну с роем, что в вашем положении приведёт к большим проблемам. Очень, очень большим. Я не зря рассказала о наших возможностях.

— Ни о чём подобном речи не идёт и не может идти.

— Знаю. Вы, пони, достаточно честны и порядочны. Поскольку я действительно пришла лишь поговорить, у меня нет оснований сомневаться в вашей порядочности. Со своей стороны, я честно предупредила. 

— Какие же требования вы хотите выдвинуть?

— Очень простые. Дань.

— Дань? В какой форме и размерах?

— О, здесь немного необычно, — вздохнула Кризалис. — Нужно кое-что ещё объяснить. Видите ли, мы питаемся чужими положительными эмоциями, прежде всего любовью, но «питаемся» не совсем точное выражение… Материальная пища поддерживает наши тела, а это… Можно провести такую аналогию: если хороший музыкант вдруг оглохнет, это не помешает ему жить, но без музыки его новая жизнь будет лишь жалким подобием предыдущей. Понимаете, о чём я?

— Кажется, да. И вы требуете дань любовью? Положительными эмоциями?

— Именно. Вы, пони, довольно весёлый и счастливый народ, этого у вас не отнимешь.

— Как же вы себе представляете такую дань?

— Это подлежит обсуждению, у нас же сейчас лишь предварительный разговор. Но вот, например, ваши праздники. Вы любите их устраивать… — при этих словах Пинки Пай радостно закивала, — и они прямо-таки лучатся положительными эмоциями. Это первое, что приходит в голову, наверняка можно придумать и что-то ещё.

— Но если речь о том, чтобы вы могли приходить на наши праздники, то тут и обсуждать нечего. Вы просто могли бы делать это в обличье обычных пони, и никто ничего не заметил бы. Тем более, не возразил бы.

— Вы ещё не понимаете. Я же сказала, что мы питаемся эмоциями. Пусть это не совсем точное слово, но когда чем-то питаются, этого становится меньше.

— Вот как?

— Да. Я не зря назвала это данью. Как видите, я честна с вами. А теперь меня интересует ваше предварительное мнение.

— Зачем оно вам, Кризалис? Вы что-то не договариваете.

— Хорошо, буду честной до конца. По этой предварительной реакции я хочу лучше оценить ваше положение, чтобы как можно больше выторговать на окончательных переговорах.

— Как вы понимаете, даже предварительное мнение мы выскажем, лишь подумав и посовещавшись.

— Конечно, — Кризалис встала. Рэрити сделала приглашающий жест и исчезла вместе с ней за дверью.

— Ну? — коротко поинтересовалась Твайлайт, когда она вернулась.

— Бред! — сказала как отрезала Эпплджек.

— К сожалению, не бред. Она по-своему абсолютно логична. Есть некое равновесие между двумя сторонами, потом одна сторона оказывается в дыре. Другой же стороне просто грех не воспользоваться этим, чтоб отхватить себе кусок побольше.

— Ну так бред и есть! Кто ж так дела делает, ты ж ей сразу копытом ткнула: назавтра в дыре-то может и кто другой оказаться!

— Это не ново. «После нас хоть потоп», слышала? Наша гостья, похоже, из таких.

— Да хоть бы и так! Всё одно бред выходит. Она себе как это вообще представляет? Вот, значит, пришла к её улью, или где они там живут, куча пони дань платить. Выстроились, развернулись в нужную сторону, и ка-ак возлюбили все разом! Любить, это всё ж таки не… — Эпплджек покосилась в сторону Рэрити и не стала говорить последнее слово.

— Можно мне?.. — тихонько сказала Флаттершай. — Странно другое. Почему она думает, что если любить или радоваться в их присутствии, то от этого любви или радости станет меньше?

— Флатти, — воскликнула Рэрити, — а я ведь, кажется, понимаю! Она же сказала, что чейнджлинги питаются эмоциями? Есть какая-то граница… ну, как между понятиями «кушать» и «жрать». Вот представьте себе повара в ресторане. Что-то он приготовил, это унесли, а потом он из кухни выглянул посмотреть в зал. Если заказавший кушает… ну, то есть наслаждается вкусом, не торопится, красиво и ловко пользуется прибором, и видно, что ему нравится… тогда повар вернётся на кухню в хорошем настроении и следующее блюдо приготовит ещё вкуснее. А если заказавший жрёт… мордой прямо в тарелку, чавкает, разбрызгивает всё вокруг себя… тут уж какое настроение! Вернётся повар назад и будет до конца дня готовить тяп-ляп.

— Интересная теория, — задумчиво проговорила Твайлайт. — То есть ты считаешь, что наша… хм… гостья?..

— Гораздо ближе к тому, чтобы жрать, чем к тому, чтобы кушать. Эпплджек, подскажи что-нибудь такое? Ещё не «жрать», но уже близкое к этому?

— Лопать? Хавать?

— Да, примерно что-то в этом роде. Ой…

Рэрити вздрогнула, и одновременно с ней вздрогнула Твайлайт.

— Чего? — встревоженно спросила Эпплджек.

— Сильный магический всплеск в… гостевых апартаментах!!!

Все пятеро выскочили из комнаты так, что чуть не застряли в дверях — после случая с Кристальным посольством Твайлайт не поленилась прочесть подругам небольшую лекцию на тему «что такое дипломатический скандал и чем это может кончиться».

Гостевые апартаменты располагались недалеко, всего через пару комнат по коридору. Своих дверей они не лишились лишь потому, что Рэрити на бегу предусмотрительно распахнула их телекинезом.

На диване рыдало ослепительно прекрасное существо, а рядом с ним валялась какая-то книжка. Скандал, похоже, всё-таки откладывался.

Твайлайт обошла вокруг дивана, рассматривая существо со всех сторон. В нём определённо было много общего с Кризалис — рост, прозрачные стрекозиные крылья, большие глаза с необычными зрачками — но цвет?! Пропорции?! Изящество?!

Существо было нежно-кремовым с переходом в тёплые жёлтые и оранжевые тона. Грива и хвост остались зелёными, но потеряли кислотный оттенок и ближе к кончикам волос тоже плавно уходили в желтизну. Крылья отливали фиолетовым.

Общие пропорции стали удивительно соразмерными, а дырчатость и зазубренность полностью исчезли. В общем, появись такое существо на каком-нибудь конкурсе красоты — и конкурс можно было бы немедленно закрывать из-за полной очевидности результата.

— Ваше величество… э… Кризалис? — позвала Твайлайт.

Существо всхлипнуло и кивнуло. Очевидно, это действительно была королева чейнджлингов. Возможность изменения облика уже не очень удивляла, но это сочетание нынешнего поведения с предыдущим…

— Ничего не понимаю! — призналась вслух Твайлайт и подцепила с дивана телекинезом книжку. — «Большой босс для Малинки»?! — она ухватила с тумбочки ещё несколько книг. — «Настоящие, или У страсти на поводу»? «Мой идеальный незнакомец»? «Одно дыхание на двоих»?! Что за… Рэрити?!..

— Да это ж бабские романы!.. — захохотала было Пинки Пай, но под выразительным взглядом Твайлайт осеклась и умолкла.

— Любовные! — с достоинством выпрямилась Рэрити. — Уже когда шли по дворцу, я услышала что-то краем уха о переговорах насчёт любви, забежала вперёд и положила здесь несколько подходящих книжек\footnote{Все упомянутые книжки с такими названиями существуют в действительности. Ничего придумывать не пришлось.}. Гостей надо принимать соответствующе. А потом проводила её сюда из переговорной, предложила устраиваться поудобнее и занять себя чтением. Сказала, что если книжки понравятся, а времени дочитать не будет, то можно забрать с собой. Всё как положено!

— Ничего не понимаю, — беспомощно повторила Твайлайт.

— Я и читала… — снова всхлипнула Кризалис. — И мне… её… стало так жа-а-алко…

— А до меня, кажись, дошло, — хмыкнула Эпплджек. — Это как в том примере, что давеча Рэрити приводила, ну про повара. Похоже, что её Номер Два, ну который жрец, сам решил покухарить, и ему оно понравилось.

— Ты что, тоже такое читаешь? — полюбопытствовала Пинки. — Ой, смотрите!.. — она ткнула копытом в сторону Рэрити. — Элемент… Элемент Щедрости? Книжек чейнджлингам не пожалела…

Теперь уже Флаттершай посмотрела на неё своим фирменным взглядом, а потом молча села рядом с Кризалис и обняла её крылом. Тихонько погладила.

— Какая же я была дура! Всё так просто… — всхлипнула та. — Вы себе не представляете, как мне теперь стыдно за то, что я вам тут наговорила!..

Твайлайт мельком глянула в указанную Пинки сторону. Действительно, на груди Рэрити висел ромбовидный фиолетовый кристалл, идеально подходящий к её метке. Но при всей странности обстоятельств, лимит удивления на сегодня был уже исчерпан.

— Э… Кризалис? Если это на вас так действует… может быть, мы выплатим вам дань, или называйте это как хотите, подобными книгами? У нас такого, хм, чтения более чем достаточно и нетрудно распорядиться написать ещё…

— Да нет же, — шмыгнула носом королева чейнджлингов. — Дело не в книгах, нужно просто пожалеть или полюбить кого-то другого… Я же говорю, это оказалось так просто… Умоляю, забудьте всё сказанное про дань, это я ваш должник. То есть, конечно, если вы нам такое дадите, это тоже будет очень кстати…

— Если вы позволите совет… Может быть, примете у себя нашего посла, или консула, или представителя, называйте это как хотите? Я знаю кое-кого, кто идеально подходит именно для возникшей сейчас ситуации…

— Ох, конечно же, если вы так считаете…

— Кто? — коротко спросила Эпплджек.

— Ты её не знаешь. Кэйденс, моя бывшая няня… у неё совершенно уникальная особая способность, именно по части зарождения любви. Я сейчас за ней пошлю, а ты сходи в дворцовую библиотеку и распорядись там, чтобы упаковали пару ящиков такого чтива, ладно? Только сама, лично. Знаю я этих библиотекарш… на них надо как следует рявкать, чтобы перестали чваниться и забéгали. Флатти, Рэрити, посидите тут пока с нашей гостьей.

\begin{center}* * *\end{center}

— Я чувствую себя как последняя дура, — вздохнула Рэрити парой часов позже, когда всё было обговорено, улажено, и Кризалис отбыла восвояси в сопровождении Кэйденс.

— Почему? Ты же получила свой Элемент. У меня вон до сих пор нету, и у Дэш нету…

— Получила, ага. Вот вернётся послезавтра Найт, пойдём мы к ней об этом докладывать, и что доложим? Эпплджек своей честностью подружила нас с Кристальной Империей. Флатти из-за обиженных детей не побоялась на саму Найтмэр Мун наорать. Пинки вынесла смехом двух Древних за одно только утро. А я? Элемент Щедрости, пф! Расщедрилась на дюжину бабских романов для какой-то жуко-пони-оборотнихи…

— Нет, — вздохнула Твайлайт. — Не так.

— А как?

— Насколько я понимаю… дело не в бабских романах и не в их количестве, дело в том, что ты её сумела толкнуть. К мысли о том, что делиться — это здорово. Любовью или чем там ещё. И она тут же в этом убедилась.

— Случайно же, из-за одного слова всё вышло.

— Не случайно. Потому что мысль «делиться здорово»… это только во-первых.

— А во-вторых?

— Во-вторых, вспомни, какая красота из этого получилась. А красота у нас уж точно по твоей части. В любом случае, тут всё далеко не так просто, как кажется.

— Интересная логика. Ну, может быть. Кстати о красоте… я тут подумала, и сейчас скажу чудовищную ересь… но по-моему, к такой внешности даже и нарядов никаких не нужно…

\chapter{Дверь и окна}

«Да как у него так получается-то?! — раздражённо подумала Рэйнбоу Дэш. — Пятнадцать минут подряд уже бубнит без остановки, и мало того, что ничего толком за это время не сказал, так ведь даже дыхание не переводил!»

Тем временем принц Блюблад продолжал разливаться соловьём, рождая сложносочинённые и сложноподчинённые перлы насчёт неустанного повышения качества мониторинга бдительности и сознательности в критических условиях всестороннего кризиса.

Сегодняшнее заседание, похоже, имело все шансы войти в историю, и не только своей глупостью. Титул хоть формально и позволял Блюбладу требовать созыва Совета, но… это был первый и пока единственный случай, когда принца послала куда подальше дворцовая стража. Твайлайт за день вымоталась настолько, что начальник караула по собственному усмотрению решил вопрос о её участии отрицательным образом, задал Блюбладу довольно замысловатый курс и выразил полную готовность пенять на себя по усмотрению Её Высочества, когда она вернётся. Подчинённые поддержали его и тоже согласились пенять на себя.

Без Найт (срочно отправившейся по какому-то вопросу к фестралам, что-то она зачастила со своими отлучками!) и отдыхавшей Твайлайт состав Совета был неполным, но всё же большинство наличествовало.

Дэш чувствовала, что быстро тупеет. Поначалу она помалкивала, не сомневаясь, что Эпплджек вот-вот выскажется про всю эту пургу (и сделает это куда лучше неё), но та почему-то сидела и спокойно слушала, хотя на приснопамятном дипломатическом приёме её терпение лопнуло гораздо быстрее.

Кажется, пора было брать ситуацию в свои копыта…

В дверь тихонько проскользнула Флаттершай, задержавшаяся на каком-то другом заседании по вопросам экологии, и Блюблад неожиданно закруглился:

— …Итак, я полагаю, все мы уже прониклись проблемой неустанного совершенствования. Как вы, наверное, догадались, собрал я вас вовсе не для этого, а лишь для того, чтобы поделиться одной прелюбопытной идеей…

Принц таки перевёл дыхание и даже глотнул воды. Переложил перед собой на столе несколько бумажек.

— Вы же помните то выездное заседание в Понивилле? Ну, по вопросам реформы образования. В одном из ваших тогдашних донесений…

\textit{«Какого сена?!»
}

— …я наткнулся на следующие слова, сказанные моей тётушкой. Чтобы не соврать… — Блюблад поднял бумажку к глазам, — «будь у вас хотя бы два Элемента из шести, и бой был бы уже на равных. А против трёх я бы без вариантов проиграла…» Вы понимаете, о чём я? В настоящий момент мы имеем Элементы Гармонии в количестве \textit{четырёх} штук.

— Четыре штуки Элементов. И ноль штук Твайлайт, — скептически заметила Рэрити. — Или у нас есть кем её заменить?

— Вопрос времени, не более… — Блюблад сделал небрежный жест копытом. — Пусть сегодня она здесь отсутствует, но мне удалось вписаться на послезавтра в её график… и я сделаю ей хорошее предложение. Очень хорошее, она крайне ценный специалист.

— Никакое предложение не гарантирует…

— У меня есть почти два дня, чтобы поработать над этим. Как вы понимаете, я буду очень стараться. А если нет… что ж, тогда… Всё-таки четырьмя Элементами мы с вами уже располагаем, и сам я не худший из магов, смею вас заверить.

— Говоря о магах, есть ещё одна проблема, — тихо сказала Флаттершай. — Солнце. Мы там как раз сейчас говорили… суточный ритм только-только приобрёл хоть какую-то определённость… Это самое слабое место всего плана.

— \textit{Узкое}, советник! Узкое, а отнюдь не \textit{слабое}. Разумеется, я думал об этом. Давайте вспомним историю: были же времена, когда светила управлялись коллективными усилиями единорогов.

— Которых требовалось для этого огромное количество! — хмыкнула Рэрити. — И каждый такой восход истощал силы всех его участников на несколько недель вперёд!

— Прекрасно, историю вы помните. А теперь ещё вспомните, что вы родом из Понивилля. В окрестностях которого имеется некий Зеркальный Пруд.

Кто-то присвистнул. Рэрити, однако, не собиралась сдаваться:

— Пусть меня поправят, если я ошибаюсь, но копия пони из Зеркального Пруда всегда получается слабее своего оригинала.

— Агась! — подтвердила Пинки Пай. — Я как-то раз… ой, только никому не говорите…

— Ну и что? — улыбнулся Блюблад. — Пусть слабее. Так ли это существенно, если копий можно наделать много?

— Копий-то наделать можно и много, — заметила Эпплджек. — А вот пруда-то при этом намного ли хватит, а?

— Хороший вопрос, советник. Тут у нас под боком есть Школа одарённых единорогов, и они всегда гордились своей научной работой. Я сунул им приличный грант и озадачил посчитать, тётушка сдуру ещё и похвалила. Расчёты уточняются, но уже сейчас выходит, что ресурсов пруда хватит на несколько десятков лет как минимум.

— А потом?

— Я озадачил историков и фольклористов. На данный момент найдены упоминания о двух других подобных локациях в Эквестрии. Поисковые экспедиции готовятся к отправке.

— Пусть ещё два пруда. Ещё несколько десятков лет. А потом?

— Наука не стоит на месте. Верю, что наши потомки решат эту проблему.

Рэйнбоу Дэш почувствовала, что её мозги идут вразнос. Мало того, что кто-то из подруг стучал — тут, похоже, стучали \textit{все}! Даже почему-то Эпплджек, которая Элемент Честности… хотя и это понять можно: если она честно верила, что так лучше, то честно и стучала. Но ведь тут не только стучали — тут ещё и действовали! Собирались действовать, во всяком случае. На полном серьёзе!

— Какого сена?! — радужная пегаска вскочила со своего места и запрыгнула на стол.

— А, советник Рэйнбоу Дэш… — кивнул принц. — Я так понимаю, вы смогли самостоятельно войти в курс дела? Имеете вопросы?

— Иметь будут тебя! А вы-то чего уши развесили?! Да как вы вообще могли пойти за этим бараном, которого даже стража посылает?!.. — и Дэш выразительно послала Блюблада от себя лично.

Флаттершай и Рэрити залились краской. Эпплджек крякнула и почесала шляпу. Пинки одобрительно присвистнула и подняла табличку с цифрами «5.7». А Блюблад и ухом не повёл.

— Затруднение… — скучным голосом констатировал он. — Придётся устранять…

Его рог засветился, но по быстроте и скорости реакции ему до Рэйнбоу Дэш было куда как далеко. А уж летать она умела намного лучше, чем ругаться.

Энергичный взлёт со стола в один взмах. Мгновенный разворот на месте. Задним копытом с разворота в лоб принцу (заодно и толчок для набора скорости). Форсаж!

Блюблад по красивой дуге улетел в угол и там прилёг, а Рэйнбоу Дэш покинула помещение через закрытую дверь (после чего помещение больше не могло похвастаться наличием двери) и со всех крыльев рванула по коридору.

Что делать дальше?!

Сегодняшняя стража, очевидно, должна быть лояльной, иначе бы этого поганца не отшили. Вот только в коридоре, как назло, ни одного гвардейца почему-то не было… хотя очень даже понятно почему! Поганец явно хорошо подумал, где именно назначать своё совещание…

Найт сейчас в отлучке и слишком далеко…

Твайлайт небось давно вырубилась и спит без задних ног… Твайлайт! Она сейчас точно не с ними, мозгов-то у неё побольше будет, да и магии тоже, а во всём Кантерлоте она нынче за главную, стража её послушается без вопросов!

Дэш выбила собой окно и выскочила наружу. Так оно быстрее, чем по коридорам, да и под открытым небом её остановить в разы сложнее. Почти половину дворца облететь нужно… вот ведь поганец-то!.. К счастью, за ней пока никто не гнался.

Найдя окно покоев Твайлайт, пегаска выбила собой и его. Лишний шум сейчас был только на пользу, проще будет разбудить подругу.

— Твай! Твайлайт!! Просыпайся, тревога!!! Я тебе должна кое-что сказать!..

\begin{center}* * *\end{center}

Твайлайт села в кровати, сонно встряхнула головой. Но выслушать поток компромата не успела.

— Нет. Это я должна кое-что сказать.

Через дверь в спальню шагнула Найт, и теперь уже Дэш встряхнула головой:

— А… э… вы… но… фестралы?..

— Была. Уже успела вернуться. Я должна извиниться перед тобой, Рэйнбоу Дэш. За то, что навела на тебя этот сон, и за то, что внутри этого сна заставила тебя усомниться в твоих подругах.

— СОН?!

— Посмотри на себя.

Дэш посмотрела и увидела на себе пижаму. Которая во многих местах висела ленточной бахромой — привет от выбитых стёкол.

— Сон. Это был обалденно реальный сон…

— Спасибо. За эту тысячу лет у меня не было других развлечений, кроме как упражняться со снами. Вот и пригодилось.

— Сон. И когда я проснулась?

— Когда выбила дверь своей спальни и выскочила в коридор. Я… не рассчитывала на такую бурную реакцию.

— Сон. У него точно нет основы? Всё звучало так логично…

— Я вам тут не очень мешаю? — осведомилась со своей кровати Твайлайт.

— Сиди и слушай, тебя это тоже касается. Я постаралась, чтобы звучало логично, но никакой основы нет. Этот ваш Блюблад…

— Почему сразу «наш»?!

— Потому что в моё время \textit{такое} при дворе не держали. Он слишком умён для того, чтобы играть против меня в заговоры — и слишком глуп, чтобы додуматься до такой интриги. Я его про себя называю «семнадцатиугольный дурак».

— Почему? — в один голос спросили Дэш и Твайлайт.

— Почему дурак… ну, сами же понимаете. А до круглого дурака он всё-таки не дотягивает, хотя и близок к этому.

— Нет, почему именно \textit{семнадцатиугольный}?

— А, это. Я же говорю, близок к \textit{круглому} дураку. Семнадцатиугольник можно построить чертёжными инструментами\footnote{Правильный семнадцатиугольник действительно можно построить циркулем и линейкой, в нашем мире это доказал Карл
Фридрих Гаусс. Судя по всему, принцесса в изгнании развлекалась ещё и математикой, благо для этого ничего, кроме чистого
разума, не требуется.}, и Блюблада тоже можно построить… неважно. Ещё раз: прими, что это был просто сон, что у этого сна никакой основы нет и что я об этом сне очень сожалею.

— Зачем тогда было его… на меня?..

— А я знаю, — хмыкнула Твайлайт. — Сейчас тебе скажут, что \textit{так было надо}.

— Да, так было надо. Но я объясню… хотя Твайлайт наверняка сама уже догадалась.

— Ага.

— Ответ у тебя на груди, Элемент Верности.

— Э… ого. Вау, молния! Круто.

Рубин был огранён косым зигзагом. Найт вздохнула:

— Я подтолкнула тебя к нему. Как могла. Понятия не имела, сработает ли это во сне и можно ли это сделать вообще… оказалось, что да. Нам очень с этим повезло. С тобой тоже: тот факт, что ты даже во сне рванулась бороться с заговором, выбив при этом дверь и окна в реальности, говорит о многом.

— А зачем было подталкивать?

— Нам нужны \textit{все} Элементы. И теперь, когда мы их имеем…

— Секунду, а шестой как же?

— Шестой должен быть твоим, Твайлайт. Элемент Магии… ну, это и с самого начала было понятно. Он особенный, его можно получить только при использовании пяти базовых. \textit{Успешном} использовании, конечно.

— А подробнее можно?

— Пять базовых элементов преобразуют личные качества своих носителей в чистые силы. Они весьма значительны… но не более того. Носитель Элемента Магии может собирать эти разные силы воедино. Вместе они образуют Гармонию подобно тому, как отдельные звуки сливаются в музыку. Их совместная мощь многократно превосходит их простую сумму… и я не знаю ничего, что могло бы этому всерьёз противостоять. Меня в своё время хватило лишь на три или четыре секунды.

— Поня-атно…

— В общем, не попробуешь — не узнаешь.

— И что, будем пробовать?

— Да. У нас есть пара часов.

— А разве время поджимает?

— И ещё как. Почему, по-твоему, я подтолкнула твою подругу? У нас проблема. Очень серьёзная проблема.

— Какая?! — Твайлайт подобралась.

— Сейчас мы пойдём на Солнечный балкон… только не в пижамах, конечно. Я уже послала за остальными, там и расскажу сразу всем.

\chapter{Радуга в ночи}

Этот балкон уже несколько веков считался тем самым местом, откуда принцесса Селестия поднимает Солнце. Разумеется, отсюда же она поднимала и Луну, но по вполне понятным причинам балкон называли Солнечным или Рассветным.

Для подъёма светила не было никакой необходимости поворачиваться к востоку. Если уж на то пошлó, вообще не обязательно было держать светило в поле зрения, не говоря даже о том, что во время своих высочайших визитов принцесса исполняла монарший долг откуда придётся. Но традиция — штука сильная, и спорить с ней просто бесполезно.

Найт это тоже прекрасно понимала, а потому, вступив на престол, ничего менять не стала — изменился только силуэт принцессы в утренние часы.

Сейчас на балконе стояли семеро пони.

— Буду краткой, — произнесла Найт. — Я не могу поднять Солнце.

Рэрити, Пинки Пай и Рэйнбоу Дэш ахнули. Твайлайт, Эпплджек и Флаттершай вздохнули — эти трое по роду своей деятельности что-то подобное уже предчувствовали.

— Вроде же почти наладилось всё… — пробормотала Рэрити.

— Увы. Вернувшись, я была слабой, но сначала мне удавалось держать день всё дольше и дольше. А вскоре после того понивилльского спектакля, на… Твайлайт?..

— На семнадцатый день.

— Да. На семнадцатый день что-то случилось. Мне как будто стало что-то мешать, и с каждым разом всё активнее. Я уже кое-как удерживала день на восьми часах, а вчера вообще выложилась до предела. Полностью. И это при том, что практически вошла в свою настоящую силу. Сегодня этой силы просто не хватает для восхода, вот и весь сказ.

— Хоть какие-нибудь соображения есть?

— Нет. Я надеялась, что сестра оставила записи про управление Солнцем — ничего не нашла. Пыталась вести себя так, будто ничего не происходит — но такое не скроешь. Твайлайт первая заметила, пристала с расспросами, я её отхуфболила… Что ты тогда подумала?

— Ну… знаете, есть такая дурацкая жеребячья манера поведения… Вроде как, если ничего не делать, то вдруг оно как-нибудь само рассосётся. Вот что-то подобное мне и подумалось.

— В самую точку. Я просто не знала, что тут сказать. Потом ещё было письмо от Сомбры… его уже так просто не отхуфболить. В ответном письме я поклялась, что приложу все усилия, лишь бы ситуация не стала хуже. Самое время прикладывать, хуже просто некуда.

— И… что же это за усилия?

— Как я уже говорила двоим из вас, мы теперь располагаем всеми пятью базовыми Элементами Гармонии. И у нас есть лучшая ученица моей сестры, которая может взять на себя шестой Элемент Магии. Я хочу воспользоваться всем этим, и других способов не вижу. Помогите мне. Пожалуйста.

— Дак об чём речь-то… — слегка недоуменно начала Эпплджек.

— Подожди, я ещё не договорила. Давайте начистоту. Всю эту кашу заварила я, и без меня её не было бы. Я могу только гадать, почему вы продолжаете мне помогать… очевидно же, что вы давно меня не боитесь. Но если сейчас хотя бы кто-то один из вас не захочет помочь по-настоящему — заведомо ничего не получится. Тогда не стоит даже и пытаться. Пусть лучше скажет сразу, я пойму.

— И что тогда? Или… что, если мы попробуем по-настоящему, но не получится?

— Тогда я верну свою сестру, как и говорила Твайлайт при той нашей встрече. А ещё я тогда говорила про грандиозное позорище. Твайлайт, не боишься?

— Унижение паче гордости. К тому же, нам и гордиться есть чем: мы подружились с кристальными пони и с чейнджлингами.

— И оба раза — не моя заслуга. Впрочем, да, хоть какое-то утешение. И… я решила. Как бы сейчас всё ни обернулось, я верну сестру в любом случае. Довольно этих глупостей.

— Но тогда какой смысл…

— А я понимаю, — тихо сказала Твайлайт. — Найт… то есть, тьфу, Луна… вы хотите что-то доказать?

— Меня давно не называли этим именем.

— Надо когда-то начинать… в смысле, возвращаться к нему. Для Найтмэр Мун вы слишком много просите, извиняетесь и объясняете. Видно же, что эта Найтмэр — пустышка, только непослушных детей пугать годится.

— Так было далеко не всегда. И я у неё кое-чему научилась.

— Ну и пусть, теперь-то какая разница… Так я правильно догадалась?

— Да, я хочу что-то доказать. Самой себе, пусть это и смешно.

— Ни разу не смешно! — воскликнула Рэйнбоу Дэш. — Перед другими выпендриваться любой дурак может, а себе-то всегда сложнее!

— Спасибо… Итак, что скажете? Ещё раз: во всём этом виновата только я.

— Зато это было самое потрясное приключение, какое можно представить!

— Мы поможем, — утвердительно сказала Твайлайт; она ни секунды не сомневалась, что остальные её поддержат. — Только… как?

— Откуда мне знать? Это ваше испытание и ваши качества… только вы можете решать, как ими правильно распорядиться. Я уже сказала тебе всё, что могла. Главное, не пытайтесь никому ничего доказывать, даже самим себе. Делайте что дóлжно, и будь что будет. Начну пока, а вы присоединяйтесь…

Луна шагнула к перилам балкона и отвернулась к востоку. Остальные вопросительно посмотрели на Твайлайт. Та вздохнула:

— Встаньте полукругом, чтобы я была в центре. Вот так… — она махнула копытом, показывая, что полукруг должен быть открыт в сторону Луны, и сама повернулась в её сторону. — С равными промежутками.

Пока сзади происходили шевеления и перестроения, Твайлайт пыталась подобрать подходящие слова. Рог Луны начал светиться.

— Приступаем. Вы должны как-то обратиться внутри себя к тем качествам, за которые получили свои Элементы. Захотеть проявить их… не применительно к чему-то конкретному, а вообще. Очень захотеть. Подвески преобразуют эти проявления в силы, хотя в теории можно даже и без них. Моей задачей будет собрать всё вместе…

Свечение вокруг рога Луны становилось ярче и ярче. Сама будучи сильным магом, Твайлайт понимала, какая мощь сейчас была ей задействована.

— Эпплджек, ты подружила нас с кристальными пони. Им это тоже нужно, ещё даже больше чем нам. Мы дома, а они только что вернулись домой после стольких веков. Рэрити, то же самое. Помни о том, насколько чейнджлинги красивы при солнечном свете. Флаттершай, не мне напоминать тебе о маленьких и слабых: ночью они беззащитны. Пинки, без света нет радости, веселья и смеха, ты это знаешь лучше всех. Рэйнбоу Дэш… просто будь верной своему имени. Тебя назвали в честь радуги, а её не бывает без Солнца…

Камни в подвесках засияли и осветили пол разными цветами. Вокруг Твайлайт на пять сторон легли пять её теней. Она умолкла, закрыла глаза и обратилась к своему магическому зрению.

Позади неё в воздухе висели пять маленьких разноцветных солнышек: оранжевое, розовое, голубое, фиолетовое и красное. Сейчас нужно было объединить их свет и отпустить его вперёд — к настоящему Солнцу, которое сегодня почему-то нуждалось в помощи. Но как?

Твайлайт попробовала потянуться к розовому шарику. Накрыть его своей аурой, подтянуть ближе к себе, потом сделать то же самое с другими, совместить…

Нет. Шарик будто выскальзывал. Фиолетовый повёл себя точно так же, и оранжевый тоже. Это было похоже… как если бы пытаться нанизать плавающую в масле горошину на вязальную спицу. Причём толстую и тупую спицу, а горошин пять штук, и все их требовалось нанизать единым движением.

А что, если…

Единорожка отказалась от концентрации и вместо этого раскрылась. Позвала солнышки к себе. Если они умеют себя вести, то должны же понимать?! К себе, но не для себя! Для важного дела, для других! Не силой, но просьбой!

Твайлайт не могла этого видеть, зато подруги увидели. Вокруг её рога вспыхнуло сияние и сгустилось в диадему с большим пурпурным кристаллом в форме шестиконечной звезды.

Разноцветные лучи от пяти подвесок сошлись на звезде и будто сплелись вокруг неё в клубок. Твайлайт снова сконцентрировалась и толкнула его от себя, как мяч.

Между звездой диадемы и рогом Луны протянулся ослепительный световой шнур. Рог засиял ещё ярче, и через несколько секунд с него в небо рванулась радуга.

Твайлайт сама не заметила, когда именно у неё открылись глаза, и это оказалось необычайно красиво — радуга в ночи. Не шесть сил, но одна. Не смесь цветов, но сплетение. Гармония.

Прочертив половину неба, радуга упала куда-то за горизонт на востоке, и вокруг этого места небо начало медленно светлеть. Ярче и ярче, но…

Слишком медленно! Поначалу усиливавшееся свечение вдруг замерцало и перестало распространяться.

— Ещё! Ещё чуть-чуть! — отчаянно крикнула Луна. — Ну пожалуйста!!!

Твайлайт поняла, чтó нужно сделать, но её на долю секунды опередила Рэрити, Элемент Щедрости:

— Девчонки, вспомните чейнджлингов! Когда этим делишься, оно только прибывает!

Радуга засияла с утроенной яркостью. Земля вздрогнула, и из-за горизонта буквально выскочило Солнце, едва ли не в зенит.

Луна мотнула головой, стряхивая радугу с рога. Теперь ей нужна была не сила, но умение: после такого рывка Солнце буквально болталось в небе, и она пыталась его стабилизировать.

— Твайлайт… проверь… пока… — уголком рта пробормотала принцесса.

Та поняла с полуслова, глянула на восток магическим зрением… и похолодела.

— Отдача… пошла…

— Быстро, ставь щит! Самый мощный, какой сумеешь! Не могу пока отвлечься… поделюсь силой, сколько получится…

Твайлайт торопливо засветила рог и выпустила из него энергию в виде заслона, прикрывающего балкон с востока. Через пару секунд в её заклинание влилась энергия от Луны, а ещё через несколько — и от Рэрити.

— А можно для нас, тёмных, объяснить? — попросила Эпплджек.

— Представь, что ты хотела выдернуть столб из земли за привязанный к нему канат, — пояснила наименее сконцентрированная сейчас Рэрити. — Очень сильно тянула и выдернула. И потом в тебя прилетает либо столбом, либо канатом. В нас сейчас тоже прилетит… чем-нибудь…

— А ежли эти Элементы опять?..

— Никакие заклинания и артефакты не работают против последствий собственного применения. Флатти, Дэши, улетайте! Ещё успеете.

Рэйнбоу Дэш фыркнула и повертела копытом возле уха. Флаттершай отрицательно помотала головой. Рэрити пожала плечами — собственно, она и не сомневалась в подобном ответе, но предложить было нужно…

— Вот как-то так… — пробормотала Твайлайт. Перед балконом теперь висела в воздухе чуть мутноватая переливающаяся плёнка вроде мыльного пузыря.

— Ого, ты постаралась… — оценила Рэрити.

— Не без твоей помощи. А толку-то? По сравнению с тем, сколько мы сейчас вбухали в восход…

— Должен выдержать… — проговорила сквозь зубы Луна, всё ещё занятая последними манипуляциями со светилом. — Хорошо поставлен, и не всё же оно назад прилетит… Канатом, а не столбом…

— Ещё секунд десять, — тихо прикинула Твайлайт, но теперь и все остальные начали что-то чувствовать, не только единороги. — Вот сейчас…

Магическая отдача грянула в щит.


\chapter{Круто попала}

Щит выполнил свою задачу. Хоть он и разрушился от удара, но поглотил при этом практически всю энергию отдачи — оставшейся хватило лишь на то, чтобы слегка покачнуть башню, сбить с ног стоявших на балконе и поднять в воздух клубы пыли.

Только что наступившее утро огласилось чиханием, кашлем и невнятными ругательствами, потом чистюля Рэрити каким-то хитрым заклинанием ликвидировала всю пыль.

На середине балкона обнаружилась раскорячившаяся на животе принцесса Селестия, намертво вцепившаяся в пол всеми четырьмя копытами.

Принцесса резко вздёрнула голову и обвела балкон каким-то полубезумным взглядом. Заметив Твайлайт, она вздохнула и ощутимо расслабилась.

— Сила есть, ума не надо… — смущённо пробормотала Селестия, села и начала подчёркнуто старательно отряхиваться.

— Рэрити, — сказала Твайлайт куда-то в пространство, — получив свой Элемент, ты сказала, что чувствуешь себя последней дурой. Помнишь?

— Ну?!

— Если тебе от этого легче, я теперь тоже.

— И я, — присоединилась Луна.

— А уж я-то… — вздохнула Селестия.

— Не соблаговолишь ли ты, сестра, — ледяным тоном произнесла Луна, — пояснить нам своё загадочное высказывание?

— Вам ещё пояснять?! Сначала аж на Луну меня выгнали, как будто нельзя было по-простому намылить мне холку, накрутить хвост и поставить в угол, если уж кое у кого копыта так чесались. Я бы даже не сопротивлялась, честное слово… Потом Солнце дёрнули так, что меня чуть не порвало. Одной только отдачи хватило на то, чтобы разбить «Великое Изгнание», я вообще не думала, что такое возможно! Потом меня об этот щит чуть не размазало, я его еле-еле развеять успела. Говорю же — сила есть, ума не надо…

— Минуточку! — в Твайлайт очень вовремя проснулся интерес исследователя. — «Рванули Солнце так, что чуть не порвало», но ведь это же означает, что…

— Это и означает. Когда кое-кто ударил по мне «Великим Изгнанием»… кстати, ума не приложу, откуда для этого взялись силы после тысячелетнего безделья…

— Всего лишь хорошо выдержанная мотивация, сестра, — ядовито сообщила Луна.

— Ещё раз: сила есть, ума не надо? Тебе стоило уделить больше внимания третьей составляющей.

— Кхм! — напомнила о себе Твайлайт.

— Да-да, запоминай на будущее, вдруг пригодится. Заклинание Великого Изгнания содержит три компоненты. Первая помещает изгоняемого в заданное место… это самое простое, остальные две много сложнее. Вторая компонента блокирует изгнанному всякий доступ на всех уровнях к тому месту, откуда изгоняли. Третья привязывает изгнанного к месту изгнания. В моём случае третья компонента имела большую слабину, и вместо изгнания получилась высылка. Впрочем, тысячу лет назад я сама допустила слабину во второй компоненте, а ведь у меня тогда была вся сила Элементов…

— Не так-то легко закрыть мне доступ к снам пони! — хмыкнула Луна.

— Да… Итак, получилась высылка. Когда через две с небольшим недели привязка развеялась, я перебралась на Солнце…

— Так там можно существовать? — не удержалась Твайлайт.

— Для меня — да. Это же всё-таки моё светило, мы с ним как две части одного целого. Там мне было уютнее… и там я проиграла пари, которое заключила сама с собой.

— И сколько же времени ты мне отводила, сестра?

— Четыре недели самое большее.

— Пф! И это после тысячи лет ожидания?!

— А сама-то ты о каком сроке думала?

— Полтора-два года.

— Столько ты бы точно не выдержала.

— Понадобилось бы — выдержала бы! А уж из принципа! А уж если бы кое-кто в своей несказанной государственной мудрости догадался оставить заметки про управление Солнцем…

— Скажи ещё — завещание.

— Смотри, как бы оно тебе и впрямь не понадобилось! Я так понимаю, это ж ты мне пыталась заклинить Солнце по утрам?!

— Ну, я… — бормотнула Селестия. — Оттуда это несложно.

— Зачем?!

— Домой хотела, знаешь ли!

— А менее эгоистичные способы тебе в голову не приходили?

— Какие, например? Твайлайт, а ты слушай, слушай, вдруг и правда пригодится!

— Могла бы просто написать на лунном диске… что-нибудь неприличное. Я бы тут же явилась прибираться и разбираться.

На физиономии Селестии отразилась досада и чёрная зависть. Твайлайт хмыкнула, оценив идею.

— Круто! — восхитилась Пинки. — А сами-то вы… чего же?

— Выставлять себя не только злодейкой, а ещё и дурой? Это же моё светило! Перестали бы его выкатывать на ночь, только и всего. Кому от этого стало бы хуже? Десятку чокнутых поэтов? Навигаторам, которые как раз всегда ценили ночь? А мне — ещё бóльшая скука.

— Ах, какое благоразумие и благородство! — фыркнула Селестия. — Раз уж мы прояснили этот вопрос, нельзя ли прояснить мне ещё кое-что? Например, с какой стати моя верная ученица помогает тут свергнувшей меня узурпаторше?

Твайлайт набрала побольше воздуха и без запинки выдала давно заготовленное:

— Причины сего суть примерно таковы же, как и те, по коим пресветлая моя наставница не удосужилась включить в свои многомудрые планы очевидный вариант того, что любимая её сестра может вернуться в здравом уме.

Луна одобрительно хмыкнула и подмигнула.

— Ну, я потом уже поняла… — смущённо призналась Селестия. — Когда вы начали Солнце тягать и сутки выстраивать. То есть о вечной ночи речь определённо не шла… а там и остальное понятно стало.

— Сие рассуждение поистине делает честь твоей сообразительности, сестра.

— А не могли бы вы двое оставить сию превыспреннюю манеру речи? Каковая даже и во время óно звучала излишне пафосно, не говоря уже о временах нынешних.

— О, конечно, конечно, — легко согласилась Луна. — Нетрудно догадаться, что за какое-то смешное тысячелетнее изгнание, которым осчастливила меня любимая сестра, многое должно было измениться.

— Лучше рассказывайте, что тут изменилось без меня! Вряд ли кое-кто догадался, что некоторые наши общие заклинания из старых времён требуют постоянного присмотра.

— Ну, я потом уже поняла, — саркастически сообщила Луна. — Когда Дискорд вылупился из статуи, а до него у нас появился новый сосед на севере.

— Сомбра?!..

— Его наследник оказался вполне вменяемым. Нам удалось подписать с ним договор о дружбе, так что это заклинание можно забыть за ненадобностью.

— А что с Дискордом?

— Даже его чувство юмора… скажем так, не выдержало некоторых аспектов нашего с тобой инцидента. Ты найдёшь его немного изменившимся, но за исключением позы, всё осталось как было.

— Что-нибудь ещё, о чём мне следует знать?

— Чейнджлинги.

— Кризалис… — поморщилась Селестия. — Могла бы и сама догадаться, уж эта в каждую бочку затычкой. Не берите в голову, я её поставлю на место. Она по-своему довольно честная стерва, и раз уж я вернулась, то все договорённости тоже возвращаются к прежнему состоянию…

— Это вряд ли, — с лёгким злорадством сообщила Луна. — Для начала, ты её при встрече буквально не узнаешь.

— Что ты с ней сделала?

— Лично я — ничего. Но бабские романы в умелых копытах буквально творят чудеса дружбы. Увидишь, тебе понравится.

— Дальше.

— Этот советник Нэйсэй… на должность его ставила ты?

— Ну, я.

— Ты ставила, ты и расколдовывай. Кстати, должность вакантна, он её занимать больше не захочет.

— А с ним ты что сделала?

Луна изобразила копытом некий жест, который ничего не говорил Твайлайт. Зато Селестия прекрасно поняла и хрюкнула от смеха:

— Безоговорочно одобряю, пусть пока побудет так. Дальше.

— Небольшая реформа системы образования.

— Я уже сама догадалась. Дальше.

— Этот твой принц Блюэ, то есть Блюблад, сейчас в Вечнодиком лесу. Руководит строительством базы аварийного обеспечения восходов.

— Надо же, к чему-то сгодился. Дальше.

— Вроде всё. Больше ничего не успели, извини. Если интересуют детали, это тебе к премьер-министру.

— Кто?

— Да вот стоит. Твайлайт же, тобою и выученная.

Солнечная принцесса прищурилась:

— Так это что?.. Вы хотите сказать, меня впервые за тысячу лет отправили в нормальный отпуск, а я, как последняя дура, примчалась обратно без всякой необходимости?!

— Вообще-то это не мы, это ты сама сейчас сказала. Вон, Элемент Честности не даст соврать.

— Я тут немножечко повторюсь, но вот как-то Элемент Верности и переход на службу к узурпатору не очень согласуются…

— Не согласуются. Просто они получили Элементы, когда ты уже была в отпуске.

— Какого сена?! А раньше, что ли, не могли?! Вовремя?

— Виновата, — пожала плечами Твайлайт. — Мне чётко поставили задачу, толком всё объяснили…

— Тьфу! — Селестия и впрямь некультурно плюнула. — И что теперь?

— Теперь? — плечами, в свою очередь, пожала Луна. — Теперь, раз уж ты вернулась, слово Пинки Пай.

Та оживилась и вытащила откуда-то из-за спины стопку исписанных бумажных листов.

— Это что?

— Это? Сценарий. Праздник победоносного возвращения принцессы Селестии из… гм… ну, теперь, видимо, из отпуска. Я в нём лично определила ряд ключевых моментов.

— А если мне не понравится и я не захочу участвовать?

— Там такой вариант тоже предусмотрен. Поверь, он понравится тебе ещё меньше.

— Встречные предложения принимаются?

— Какие же?

— Я… э… не против отпуска. Это, конечно, задним числом… но не против.

— Всегда пожалуйста. Тебя обратно на Луну?

— Говоря о Луне… тебе всегда удавалась красота ночи. Это тоже задним числом, но тут мне до тебя далеко.

— Истину глаголишь.

— Ну и вот. Праздник устроим, конечно. Я сделаю красивый закат, ты сделаешь красивую лунную ночь. А потом будет… хм, инаугурация премьер-министра. С роскошным балом…

Глаза Рэрити заблестели.

— …птичьим оркестром…

Флаттершай оживилась.

— …авиашоу…

Теперь оживилась Рэйнбоу Дэш.

— …угощением. Твои советники наверняка смогут посоветовать хорошего поставщика продуктов…

Эпплджек задумчиво прищурилась.

— Ну, а сценарист для проработки деталей, я так понимаю, у тебя на примете уже есть.

Пинки улыбнулась до ушей и помахала копытом.

— Вот. И все довольны. Ты будешь созидать ночную красоту, я отдыхать… ну, Солнце, конечно, на мне, это понятно. А для рутины у нас теперь есть премьер-министр, это ты хорошо придумала. Как говорят в Сталлионграде, кадры решают всё. Иногда только придётся улыбаться, махать и подписывать. 

— Интере-е-есная мысль… — протянула Луна, задумчиво глядя на Твайлайт. — Кадры, говоришь? Кадры… это, конечно, хорошо. Но ведь ненадолго. А потом?

— В чём-то ты права. Ну… тогда устроим коронацию вместо инаугурации.

— Но ведь она…

— Она сильный маг. А тот задачник Старсвирла я сберегла. Помнишь?

— Ещё бы!

— Ну и вот. Будет у нас в Эквестрии три принцессы. Принцесса Дня, принцесса Ночи и…

— Принцесса Бюрократии? — съязвила Луна.

— Ну зачем же. Гармонии… это ведь почти то же самое, но звучит куда лучше. Или, скажем, Дружбы, раз вы при ней договор с кристальными подписали.

— Ещё чейнджлинги.

— Ну, тем более. Значит, будет принцессой Дружбы. Практику она у тебя прошла, в таких условиях неделю за год считать можно…

Твайлайт стояла ни жива, ни мертва. Подруги уже наперебой поздравляли её:

— Ну, ты попала! — сказала Элемент Честности.

— Круто попала, — уточнила Элемент Верности.

— Не бойся, лучше подумай, сколько хорошего теперь сможешь сделать, — тихонько проговорила Элемент Доброты.

— Для всех, — добавила Элемент Щедрости.

— Агась! И это будет прикольно! — пообещала Элемент Смеха.

Но из всех этих поздравлений в сознании будущей принцессы почему-то эхом пульсировало лишь одно.

«Круто попала…»

\chapter{Утро на объекте (альтернативная концовка)}




Режимный объект стратегического назначения «Восход-17д» был назван в честь принца Блюблада, о чём знал весь его персонал. Ещё бы не знать, если сей факт был прямо озвучен Её Высочеством в речи на церемонии открытия!

Причина тоже не была ни для кого секретом: в прозвучавшей тогда же речи Её Светлости упоминалось, что принц лично разработал некоторые идеи, лежащие в основе функционирования объекта. Это вполне согласовывалось с тем, что Блюблад руководил строительством и первые несколько недель работы сам же здесь командовал, причём командовал вполне толково.

Правда, за прошедшие годы никто так и не сумел разглядеть в названии «Восход-17д» то самое «назван в честь», и в конце концов личный состав махнул на эту загадку копытом. Было решено, что Её Высочество, известная своей любовью к старомодной манере речи, просто не вполне удачно выразилась и имела в виду лишь то, что честь выбора названия была предоставлена принцу как разработчику.

Всё это традиционно рассказывалось высоким гостям объекта в случае визитов… вот как, например, сегодня, когда была приглашена делегация яков. К таким визитам давно относились спокойно и даже слегка равнодушно: трепетать перед титулами просто разучились (разучишься тут, постоянно общаясь с Её Высочеством, Её Светлостью и Их Превосходительствами — вполне нормальные пони, хоть и с титулами, чего трепетать-то?), а показушничать вообще не имело никакого смысла. И так старались изо всех сил… для себя же старались, в конце концов!

Гостей проводили в специальный бункер, активировали для них трансляцию происходящего через сеть магических кристаллов и приступили к процедуре.

— «Жёлтая» готовность, минус пять минут! — разнеслось над объектом. — Подразделениям доложить!

— Третий образцовый отряд на дежурство заступил!

Кристалл показал Зеркальный Пруд, по берегу которого через равные промежутки стояли маги из числа лучших выпускников Школы им. Принцессы Селестии.

— Обсерватория готова!

Под куполом башни в окружении хитрой аппаратуры устроились трое единорогов. Над башней порхали двое пегасов с портативными дублирующими инструментами.

— Погодно-патрульная группа отработала, небо свободно, восток чист!

Эскадрилья пегасов-погодников в данный момент строем заходила на четвёртый разворот аэродромного круга.

— Наземный патруль в оцеплении, периметр свободен, посторонних нет!

Командир патруля земных пони улыбнулся и козырнул в кристалл. Сопровождающий в бункере торопливо пояснил гостям, что странно выглядящая и откровенно демаскирующая форма патрульных специально делает их видимыми издалека, дабы как можно нагляднее показывать гражданскому населению границы опасной зоны.

— Группа обеспечения отработала, наблюдаю группу перехвата на месте!

Кристалл показал десяток земных пони и единорогов, обвешанных плотницким инструментом — в настоящий момент они находились на помосте, сооружённом ими на вершине дерева. Плотники с грохотом запрыгали, проверяя качество работы и одновременно демонстрируя его тем, для кого помост предназначался.

— Так точно, группа перехвата на месте, эвакуирую группу обеспечения!

Плотники соскочили с помоста, пегасы из группы перехвата подхватили их, опустили на землю и снова взмыли в воздух.

— Пожарный расчёт на дежурство заступил!

Вокруг ствола со своим инвентарём кольцом разместились огнеборцы, среди которых можно было видеть представителей всех трёх рас.

— Группа прикрытия на месте!

Вокруг пожарных вторым кольцом стояли маги, тоже из числа выпускников Школы.

— Принято! — подытожил командир объекта. — Минус две минуты, восходному расчёту «зелёная» готовность, желаем удачи!

К дереву выдвинулся восходный расчёт — лично Её Высочество и шесть Элементов Гармонии во главе с Её Светлостью. Маги из группы прикрытия телепортировали их на помост (прямой необходимости в этом не было, двое из расчёта сами владели телепортацией и трое умели летать, но устав предписывал поступать именно так — дабы расчёт не отвлекался на второстепенные задачи).

— Расчёт на позиции! — отрапортовал командир «прикрышки».

— Работа по процедуре, всем действовать по обстановке!

Процедура была зрелищной (зря, что ли, смотреть на неё водили гостей), но для личного состава совершенно непонятной. Даже маги не могли ничего толком прокомментировать — используемая методика-де очень специфична, их самих ничему такому сроду не учили, описать словами затруднительно из-за отсутствия в языке подходящих слов (однажды прозвучавшее «это очень сильное колдунство» было теперь неофициальным девизом объекта)… Впрочем, последовательность действий все знали и помнили наизусть.

Сначала сверху невнятно доносились внутренние переговоры расчёта (из-за высоты ничего толком не разобрать, а подслушивать магией дураков не было). Потом там разгоралось разноцветное сияние, за которым следовала вспышка. Наконец, с помоста стартовала радуга, перекидывалась через полнеба, и этой радугой восходный расчёт вытаскивал, как удочкой, Солнце из-за горизонта.

Иногда бывало так, что Солнце застревало — тогда наверху раздавались лихорадочные команды, радуга усиливалась и восход сопровождался небольшим землетрясением. Трижды случалось, что вытащить Солнце радугой не удавалось, и тогда в дело вступал образцовый отряд. Выглядело это жутко, на весь личный состав потом проливался дождь наград, и все раз навсегда усвоили, почему эти отряды так называются — от слова «образцы»…

Но сегодня всё прошло идеально. Всего лишь несколько секунд радужного явления — и светило послушно выскочило из-за горизонта.

— Восход успешный, над горизонтом двадцать шесть, яркость норма! — скороговоркой отрапортовала обсерватория. — Отдача пошла, ноль-четыре номинала, расчётное время… тридцать одна!

Один из магов прикрытия высветил в небе иллюзию: цифры обратного отсчёта до удара отдачей. Тридцать одна, тридцать, двадцать девять секунд…

— Решение?! — потребовал командир объекта.

Наверху произошло шевеление, короткий обмен репликами, и Её Светлость коротко озвучила по трансляции:

— Сами!

Образцовый отряд у Зеркального Пруда облегчённо выдохнул. Это означало, что их работа на сегодня практически окончена: не нужно уже осуществлять «оперативное самоклонирование с последующим руководством клонами по подъёму светила традиционным способом», да и помогать «прикрышникам» ставить щит не понадобится. Нервная работёнка, что и говорить: видеть, как из пруда выходит стадо твоих точных копий и потом по сути пускать их в расход… брр!.. После учений, которые иногда приходилось проводить, ребята несколько дней были сами не свои, и их старались без крайней необходимости не трогать…

Процедура меж тем завершалась. Расчёт эвакуировался самостоятельно: элементы Честности и Щедрости были подхвачены элементами Верности и Доброты и плавно опущены на землю. (Пегасы группы перехвата сопроводили их в полёте, будучи готовыми подстраховать, но это было чистой формальностью.) Элемент Смеха с радостным воплем «йо-ххо!!!» сиганула с помоста на брезентовое полотнище, растянутое пожарным расчётом.

Двое старших, как и полагается, покинули помост последними. Её Светлость подождала, пока Её Высочество окончательно стабилизирует светило в небе, и обе они телепортировались вниз.

— Внимание!!! — раскатилось над объектом. Отсчёт в небе приближался к нулю.

Сверху завыло, застонало, и помост на вершине дерева отдачей разнесло на кучу обломков. Но это давно стало привычным и никого не впечатлило.

Крупные обломки поймала группа перехвата и отнесла в сторону: группе обеспечения предстояло осмотреть их и решить, чтó из этого завтра пойдёт на сооружение нового помоста. Мелкие обломки (будущие дрова) остановила телекинезом группа прикрытия. Загоревшуюся было крону моментально потушил несколькими меткими струями воды пожарный расчёт. Собственно, это было и всё.

Магический кристалл быстро показал гостям череду сцен. Образцовый отряд, не прекращая своего дежурства, отдыхал на берегу пруда. Погодники в штурманской проводили текущий разбор полётов. Наземные патрули возвращались на базу. Персонал обсерватории зачехлял свою хитрую технику. Вокруг дерева собрались «перехватчики», «прикрышники» и пожарные — эти делились впечатлениями и принимали благодарности от Её Высочества с Её Светлостью. Дерево уже осматривали лесники из группы сохранения, а в сторонке плотники прикидывали что-то возле обломков помоста…

Гости кивали как заведённые и восторженно повторяли «о да, очень сильный колдунство, пони уметь колдунство» (в устах яков девиз объекта звучал на удивление естественно). Адьютанты вкатывали в помещение фуршетные столики с закусками, а самый голосистый уже стоял у двери в полной готовности орать «приветствуем Её Высочество!..»

В общем, этот день восьмого года царствования Найтмэр Мун обещал быть удачным. Восход выдался просто на загляденье, гости пришли в полный восторг… а после заката по календарю полагалось новолуние, и Её Высочество собиралась впечатлить их какой-то особенно изысканной звёздной ночью собственной работы.

Благо ночное светило с неудобной надписью «{\huge Я В ОТПУСКЕ!}» по случаю новолуния выкатывать на небо не требовалось.



\end{document}
